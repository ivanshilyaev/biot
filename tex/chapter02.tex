\chapter{Безопасность сетевых протоколов IoT}

	\section{ZigBee}
	Одной из определяющих особенностей ZigBee является предоставление средства для осуществления 
	безопасной связи, защиты создания и транспортировки криптографических ключей, шифрования кадров 
	и управления устройствами. Данная особенность основывается на базовой структуре безопасности, определенной в 
	стандарте IEEE 802.15.4. Эта часть архитектуры опирается на правильное управление симметричными 
	ключами и корректную реализацию методов и политик безопасности.
	
	Основным механизмом обеспечения конфиденциальности является защита всего ключевого материала. 
	Доверие между сторонами предполагается при первоначальной установке ключей, а также при обработке 
	информации о безопасности.
	
	Проблема защиты и безопасного распределения ключей является первостепенной в любой системе
	безопасности. Ключи никогда не должны передаваться по незащищенному каналу. В случае ZigBee 
	кратковременное исключение из этого правила происходит на начальном этапе добавления в сеть 
	ранее не сконфигурированного устройства. Модель сети ZigBee уделяет особое внимание соображениям
	безопасности, поскольку сети, формирующие свою структура <<на лету>> (ad-hoc сети), могут быть 
	физически доступны для внешних устройств. Также невозможно предсказать состояние рабочей среды.
	
	В стеке протоколов различные сетевые уровни не разделены криптографически, поэтому необходимыми 
	являются политики доступа. Открытая модель доверия внутри устройства позволяет совместно использовать 
	ключи, что значительно снижает потенциальную стоимость. Если могут существовать вредоносные устройства, 
	то каждая полезная нагрузка сетевого уровня должна быть зашифрована, чтобы несанкционированный 
	трафик мог быть немедленно прерван. Исключением, опять же, является передача сетевого ключа, 
	который передаёт единый уровень безопасности сети новому присоединяющемуся устройству.
	
	ZigBee использует симметричное шифрование AES с длинной ключа 128 бит для реализации своих механизмов 
	безопасности. Ключ может быть связан либо с сетью, что позволяет использовать его обоими уровнями 
	ZigBee и подуровнем MAC, либо с каналом, полученным в результате предварительной установки, соглашения 
	или транспортировки. Создание ключей канала связи основывается на использовании главного ключа. 
	В конечном итоге, по крайней мере, главный ключ (мастер-ключ) должен быть получен через безопасную 
	среду (с помощью защищённого канала связи или по предварительной установке), поскольку от этого 
	зависит безопасность всей сети. Главный ключ виден только на прикладном уровне. Различные службы 
	используют разные односторонние вариации ключа связи, чтобы избежать утечек и рисков безопасности.
	
	В безопасной сети для распределения ключей назначается одно специальное устройство, которому доверяют 
	другие устройства: так называемый удостоверяющий центр. При идеальном сценарии устройства должны 
	иметь адрес удостоверяющего центра и начальный главный ключ, предварительно загруженный в них. 
	Типичные приложения без особых требований к безопасности будут использовать для связи данный сетевой ключ, 
	предварительно предоставленный удостоверяющим центром.
	
	Архитектура безопасности распределена между сетевыми уровнями следующим образом:
	
	\begin{itemize}
		\item Подуровень MAC (уровень управления доступом к среде) способен обеспечивать надежные соединения
		между двумя устройствами. Как правило, уровень безопасности, который он использует, задается 
		верхними уровнями.
		\item Сетевой уровень управляет маршрутизацией, обрабатывает полученные сообщения и транслирует
		запросы. Исходящие кадры будут использовать ключ соединения в соответствии с маршрутизацией, 
		если он доступен; в противном случае для защиты полезной нагрузки от внешних устройств будет 
		использоваться сетевой ключ.
		\item Прикладной уровень обеспечивает создание ключей и транспортные услуги как для объектов
		сети, так и для приложений.
	\end{itemize}

	
	\section{Z-Wave}
	До 2008 года в спецификации Z-Wave не было никаких упоминаний о способах защиты каналов связи. 
	Таким образом, все устройства Z-Wave коммуницировали открыто. Это означало, что любая сеть Z-Wave 
	была доступна извне и взламывать её было не нужно. В 2008 в спецификацию было добавлено понятие 
	шифрования (Z-Wave S0 Security), а в качестве алгоритма шифрования был выбран AES с длиной ключа 128 бит. 
	Это изменение было призвано решить проблему распространения устройств Z-Wave. Однако разработчики
	не учли мелких деталей.
	
	В 2013 году в спецификации  Z-Wave S0 Security была обнаружена уязвимость. В момент первичной 
	инициализации соединения перед началом сеанса передачи данных устройству отправляется ключ шифрования.
	На тот момент этот ключ представлял собой последовательность из 128 нулей. Таким образом, злоумышленник
	мог легко подслушать первичный сеанс связи, ключ которого был заранее известен. Далее не
	составляло труда отследить последующие изменения ключей шифрования. В результате практически
	каждая Z-Wave сеть оказалась уязвимой.
	
	После этой истории репутация Z-Wave была существенно испорчена. Для решения проблемы в 2016 году
	появилась улучшенная версии спецификации Z-Wave S2 Security. В ней для первичной выработки ключей
	используется протокол Диффи-Хеллмана.
	
	
	\section{Wi-Fi}
	С точки зрения безопасности Wi-Fi необходимо учитывать среду передачи сигнала. В беспроводных сетях 
	получить доступ к передаваемой информации и повлиять на канал передачи данных значительно проще. 
	Для этого достаточно поместить соответствующее устройство в зоне действия сети. Основными протоколами
	защиты информации в сетях Wi-Fi являются WEP, WPA, WPA2 и WPA3.
	
	Протокол WEP (Wired Equivalent Privacy) был разработан в 1997 году вместе с первой версией Wi-Fi.
	Для шифрования использовался алгоритм RC4 со статическим ключом длиной 64 или 128 бит. Некоторое время
	протокол успешно функционировал и был способен противостоять базовым атакам типа <<человек посередине>>.
	Недостатками для шифрования являлись малая длина ключа, а так же использование для шифрования 
	непосредственно пароля, предоставленного пользователем. Wi-Fi Alliance отказался от использования WEP
	в 2004 году, официально объявив этот протокол небезопасным.
	
	В 2003 году на замену WEP пришёл протокол WPA (Wi-Fi Protected Access). WPA использует протокол 
	целостности временного ключа (TKIP) с всё той же длинной ключа 64 или 128 бит. Для шифрования
	использовался тот же самый алгоритм RC4, но вектор инициализации был увеличен вдвое: с 24 до 48 бит.
	TKIP использует ключ на каждый пакет, то есть динамически генерирует новый 128-битный ключ для 
	каждого пакета и таким образом предотвращает типы атак, которые скомпрометировали WEP. Несмотря
	на все улучшения протокол WPA позиционировался как временная мера для замены уязвимого протокола
	WEP.
	
	Полноценной же заменой стал протокол WPA2, который появился в использовании с 2004 года. С этого
	же года началась сертификация, а с 2006 по 2020 год сертификация WPA2 была обязательной для всех
	новых устройств с торговой маркой Wi-Fi. В качестве замены TKIP был внедрён протокол блочного 
	шифрования с имитовставкой и режимом сцепления блоков и счётчика: CCMP. TKIP использовался
	только для обратной совместимости. CCMP основан на алгоритме шифрования AES с длиной ключа 128 бит.
	Преимуществом по сравнение со старыми версиями является генерация ключей шифрования во время
	соединения, а не их статическое распространение.
	
	В январе 2018 года Wi-Fi Alliance объявил WPA3 в качестве замены WPA2. Сертификация началась в июне 
	2018 года. Самое крупное изменение связано с новым методом аутентификации. SAE (Simultaneous 
	Authentication of Equals) явился заменой PSK (Pre-Shared Key), который использовал четырёхэтапное 
	установление связи в протоколе WPA2. SAE работает на основе предположения о равноправности 
	устройств, вместо того, чтобы считать одно устройство отправляющим запросы, а второе $-$ предоставляющим 
	право на подключение \cite{802.11-2016}.  Кроме того, SAE обеспечивает защиту от <<чтения назад>>. 128-битное шифрование
	осталось минимально допустимой нормой, а для промышленных масштабов предложен вариант использования
	AES с длиной ключа 256 бит в режиме GCM с SHA-384 в качестве HMAC для контроля целостности информации.
	С выходом WPA3 появились две новые сопутствующие технологии: Easy Connect  и Enhanced Open. Первая
	позволяет сделать процесс добавления новых устройств в сеть более простым и является весьма актуальной
	для домашней автоматизации. Отныне у каждого устройства будет уникальный QR-код, который сможет
	работать в качестве открытого ключа. Для добавления устройства необходимо будет просканировать код 
	при помощи смартфона, который уже находится в сети. После сканирования между устройством и сетью 
	произойдёт обмен ключами аутентификации для установления последующего соединения.
	Вторая технология усиливает защиту пользователей в открытых сетях, где по умолчанию нет защиты с
	аутентификацией.
	Данные технологии не зависят напрямую от WPA3, но улучшают безопасность для определённых типов сетей.
	Таким образом, вместо полной переработки безопасности Wi-Fi, WPA3 концентрируется на новых технологиях, 
	которые позволят устранить уязвимости, обнаруженные в WPA2.
	
	
	\section{Сравнение безопасности}
	Весьма схожие в техническом плане ZigBee и Z-Wave, с точки зрения безопасности также не имеют
	существенных различий: оба протокола используют симметричный алгоритм блочного шифрования
	AES с длиной ключа 128 бит. Единственным отличием является метод распределения ключей: в Z-Wave
	используется протокол Диффи-Хеллмана, в то время как в ZigBee этот процесс по-прежнему доверен
	центру управления безопасностью.
	
	В сетях на основе Wi-Fi безопасности уделяется значительно больше внимания в силу их повсеместного
	распространения. В контексте домашней автоматизации и некоторых других сфер, в которых Wi-Fi
	конкурирует с ZigBee и Z-Wave, отличия всё также незначительны. Для шифрования используются
	дополнительные надстройки и протоколы, но в основе лежит AES-128. В случае промышленного
	использования для Wi-Fi применяются более продвинутые технологии защиты информации.
	