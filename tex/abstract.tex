\newpage
\begin{center}{\bf \Large Реферат}\end{center}
  
\textbf{Дипломная работа:} 68 страниц, 4 главы, 13 рисунков, 5 таблиц, 14 использованных источников,
2 приложения.

\textbf{Ключевые слова:} ИНТЕРНЕТ ВЕЩЕЙ, КРИПТОГРАФИЧЕСКАЯ ЗАЩИТА ДАННЫХ, 
АУТЕНТИФИЦИРОВАННОЕ ШИФРОВАНИЕ.

\textbf{Объект исследования:} криптографическая защита данных в протоколах, применяемых в сфере 
интернета вещей.

\textbf{Цель работы:} изучение сетевых протоколов, применяемых в сфере интернета вещей, изучение
и сравнение безопасности этих протоколов, анализ уязвимостей и угроз, разработка прототипа умного
устройства и протокола взаимодействия с применением белорусских криптографических стандартов.

\textbf{Методы исследования:} а) теоритические: изучение источников, посвящённых протоколам,
применяемым в сфере интернета вещей; изучение характеристик этих протоколов, методов, применяемых
для защиты данных; б) практические: составление матрицы, сравнивающей устойчивость выбранных 
технологий к некоторому общему набору угроз в целях выявления наиболее криптостойкого решения;
разработка прототипа умного устройства на собственной прошивке, использующей методы защиты
данных, описанные в белорусском криптографическом стандарте.

\textbf{Результат:} разработанная прошивка, использующая sponge-функцию и аутентифицированное
шифрование, описанные в стандарте СТБ 34.101.77, а также устройство, работающее на этой прошивке.

\textbf{Область применения:} сфера информационной безопасности и интернета вещей.



\newpage
\begin{center}{\bf \Large Рэферат}\end{center}

\textbf{Дыпломная праца:} 68 старонак, 4 раздзела, 13 малюнкаў, 5 табліц, 14 выкарыстаных крыніц,
2 дадатку.

\textbf{Ключавыя словы:} ІНТЭРНЭТ РЭЧАЎ, КРЫПТАГРАФІЧНАЯ АБАРОНА ДАНЫХ, АЎТЭНТЫФІКАВАНАЕ 
ШЫФРАВАННЕ.

\textbf{Аб'ект даследавання:} крыптаграфічная абарона дадзеных у пратаколах, якія выкарыстоўваюцца 
ў сферы інтэрнэту рэчаў.

\textbf{Мэта працы:} вывучэнне сеткавых пратаколаў, якія прымяняюцца ў сферы інтэрнэту рэчаў, вывучэнне 
і параўнанне бяспекі гэтых пратаколаў, аналіз уразлівасцяў і пагроз, распрацоўка прататыпа разумнай прылады 
і пратаколу ўзаемадзеяння з ужываннем беларускіх крыптаграфічных стандартаў.

\textbf{Метады даследавання:} а) тэарытычныя: вывучэнне крыніц, прысвечаных пратаколам, якія прымяняюцца 
ў сферы інтэрнэту рэчаў; вывучэнне характарыстык гэтых пратаколаў, метадаў, якія прымяняюцца для абароны 
дадзеных; б) практычныя: складанне матрыцы, якая параўноўвае ўстойлівасць абраных тэхналогій да некаторага 
агульнага набору пагроз у мэтах выяўлення найболей крыптаўстойлівага рашэння; распрацоўка прататыпа 
разумнай прылады на ўласнай прашыўцы, якая выкарыстоўвае метады абароны дадзеных, апісаныя ў беларускім 
крыптаграфічным стандарце.

\textbf{Вынік:} распрацаваная прашыўка, якая выкарыстоўвае sponge-функцыю і аўтэнтыфікаванае шыфраванне, 
апісаныя ў стандарце СТБ 34.101.77, а таксама прылада, якая працуе на гэтай прашыўцы.

\textbf{Вобласць ужывання:} сфера інфармацыйнай бяспекі і інтэрнета рэчаў.


\newpage
\begin{center}{\bf \Large Abstract}\end{center}

\textbf{Diploma thesis:} 68 pages, 4 chapters, 13 figures, 5 tables, 14 sources, 2 attachments.

\textbf{Keywords:} INTERNET OF THINGS, CRYPTOGRAPHIC DATA \newline PROTECTION, 
AUTHENTICATED ENCRYPTION.

\textbf{Object of study:} cryptographic data protection in protocols used in the Internet of Things.

\textbf{Purpose of work:} study of networking protocols used in the Internet of Things, study and 
comparison of these protocols security, analysis of vulnerabilities and threats, development 
of a smart device prototype and interaction protocol using Belarusian cryptographic standards.

\textbf{Research methods:} a) theoretical: study of the sources devoted to protocols, used in the 
Internet of Things; study the characteristics of these protocols and data protection methods; 
b) practical: creation of matrix comparing the robustness of the selected technologies to a common 
set of threats in order to identify the most crypto-resistant solution; development 
of a smart device prototype on its own firmware, using data protection methods described in the Belarusian 
cryptographic standard.

\textbf{Result:} developed firmware that uses sponge function and authenticated encryption described 
in the standard СТБ 34.101.77, as well as a device running on this firmware.

\textbf{Scope:} the sector of information security and the Internet of Things.
