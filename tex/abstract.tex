\newpage
\begin{center}{\bf \Large Реферат}\end{center}
  
\textbf{Дипломная работа:} 69 страниц, 4 главы, 13 рисунков, 5 таблиц, 14 использованных источников,
5 приложений.

\textbf{Ключевые слова:} ИНТЕРНЕТ ВЕЩЕЙ, КРИПТОГРАФИЧЕСКАЯ ЗАЩИТА ДАННЫХ, 
АУТЕНТИФИЦИРОВАННОЕ ШИФРОВАНИЕ, КРИПТОГРАФИЧЕСКИЙ ПРОТОКОЛ,
SPONGE-ФУНКЦИЯ.

\textbf{Объект исследования:} криптографическая защита данных в протоколах, применяемых в сфере 
интернета вещей.

\textbf{Цель работы:} изучение сетевых протоколов, применяемых в сфере интернета вещей, изучение
и сравнение безопасности этих протоколов, анализ уязвимостей и угроз, разработка прототипа умного
устройства и протокола взаимодействия с применением белорусских криптографических стандартов.

\textbf{Методы исследования:} а) теоретические: изучение источников, посвящённых протоколам,
применяемым в сфере интернета вещей; изучение характеристик этих протоколов, методов, применяемых
для защиты данных; б) практические: составление матрицы, сравнивающей устойчивость выбранных 
технологий к некоторому общему набору угроз в целях выявления наиболее криптостойкого решения;
разработка прототипа умного устройства на собственной прошивке, использующей методы защиты
данных, описанные в белорусском криптографическом стандарте.

\textbf{Результат:} сравнение технических характеристик выбранных протоколов; сравнение безопасности
выбранных протоколов; описание известных угроз и успешно проведённых атак на различные версии
протоколов; построенная матрица угроз; реализация алгоритмов аутентифицированного шифрования 
и хэширования из белорусского криптографического стандарта СТБ 34.101.77 на языках программирования
Java и C++; разработанная прошивка для умного устройства на языке программирования C++
с использованием аутентифицированного шифрования; разработанный прототип умной лампочки, работающий
на этой прошивке; разработанное веб-приложение для управляющего устройства на языке
программирования Java.

\textbf{Область применения:} сфера информационной безопасности и интернета вещей.



\newpage
\begin{center}{\bf \Large Рэферат}\end{center}

\textbf{Дыпломная праца:} 69 старонак, 4 раздзела, 13 малюнкаў, 5 табліц, 14 выкарыстаных крыніц,
5 дадаткаў.

\textbf{Ключавыя словы:} ІНТЭРНЭТ РЭЧАЎ, КРЫПТАГРАФІЧНАЯ АБАРОНА ДАДЗЕНЫХ, АЎТЭНТЫФІКАВАНАЕ 
ШЫФРАВАННЕ, КРЫПТАГРАФІЧНЫ ПРАТАКОЛ, SPONGE-ФУНКЦЫЯ.

\textbf{Аб'ект даследавання:} крыптаграфічная абарона дадзеных у пратаколах, якія выкарыстоўваюцца 
ў сферы інтэрнэту рэчаў.

\textbf{Мэта працы:} вывучэнне сеткавых пратаколаў, якія прымяняюцца ў сферы інтэрнэту рэчаў, вывучэнне 
і параўнанне бяспекі гэтых пратаколаў, аналіз уразлівасцяў і пагроз, распрацоўка прататыпа разумнай прылады 
і пратаколу ўзаемадзеяння з ужываннем беларускіх крыптаграфічных стандартаў.

\textbf{Метады даследавання:} а) тэарытычныя: вывучэнне крыніц, прысвечаных пратаколам, якія прымяняюцца 
ў сферы інтэрнэту рэчаў; вывучэнне характарыстык гэтых пратаколаў, метадаў, якія прымяняюцца для абароны 
дадзеных; б) практычныя: складанне матрыцы, якая параўноўвае ўстойлівасць абраных тэхналогій да некаторага 
агульнага набору пагроз у мэтах выяўлення найболей крыптаўстойлівага рашэння; распрацоўка прататыпа 
разумнай прылады на ўласнай прашыўцы, якая выкарыстоўвае метады абароны дадзеных, апісаныя ў беларускім 
крыптаграфічным стандарце.

\textbf{Вынік:} параўнанне тэхнічных характарыстык выбраных пратаколаў; параўнанне бяспекі выбраных пратаколаў; 
апісанне вядомых пагроз і паспяхова праведзеных нападаў на розныя версіі пратаколаў; пабудаваная матрыца пагроз; 
рэалізацыя алгарытмаў аўтэнтыфікаванага шыфравання і хэшавання з беларускага крыптаграфічнага стандарту 
СТБ 34.101.77 на мовах праграміравання Java і C++; распрацаваная прашыўка для разумнай прылады на мове 
праграмавання C++ з выкарыстаннем аўтэнтыфікаванага шыфравання; распрацаваны прататып разумнай лямпачкі, 
які працуе на гэтай прашыўцы; распрацаванае вэб-прыкладанне для кліентскай прылады на мове праграмавання Java.

\textbf{Вобласць ужывання:} сфера інфармацыйнай бяспекі і інтэрнэта рэчаў.


\newpage
\begin{center}{\bf \Large Abstract}\end{center}

\textbf{Diploma thesis:} 69 pages, 4 chapters, 13 figures, 5 tables, 14 sources, 5 attachments.

\textbf{Keywords:} INTERNET OF THINGS, CRYPTOGRAPHIC DATA \newline PROTECTION, 
AUTHENTICATED ENCRYPTION, CRYPTOGRAPHIC PROTOCOL, SPONGE-FUNCTION.

\textbf{Object of study:} cryptographic data protection in protocols used in the Internet of Things.

\textbf{Purpose of work:} study of networking protocols used in the Internet of Things, study and 
comparison of these protocols security, analysis of vulnerabilities and threats, development 
of a smart device prototype and interaction protocol using Belarusian cryptographic standards.

\textbf{Research methods:} a) theoretical: study of the sources devoted to protocols, used in the 
Internet of Things; study of these protocols characteristics and data protection methods; 
b) practical: creation of matrix comparing the robustness of the selected technologies to a common 
set of threats in order to identify the most crypto-resistant solution; development 
of a smart device prototype on its own firmware, using data protection methods described in the Belarusian 
cryptographic standard.

\textbf{Result:} comparison of selected protocols technical characteristics; comparison of selected protocols security; 
description of known threats and successful attacks on different versions of protocols; constructed threat matrix; 
implementation of authenticated encryption and hashing algorithms from the Belarusian cryptographic standard СТБ 34.101.77 
in Java and C++ programming languages; developed firmware for smart device in C++ programming language using 
authenticated encryption; developed prototype of smart bulb running on this firmware; developed web application 
for client device in Java programming language.

\textbf{Scope:} the sector of information security and the Internet of Things.
