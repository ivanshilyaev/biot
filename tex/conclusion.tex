\chapter*{Заключение}
	\addcontentsline{toc}{chapter}{Заключение}
	
	Интернет вещей всё глубже проникает в жизнь конечного потребителя, набирая всё большую популярность.
	В связи с этим вопросы обеспечения безопасности конечных устройств и криптографической защиты данных
	пользователей остаются открытыми и актуальными.
	
	В первой главе текущего исследования были описаны ключевые технологии, применимые в IoT. Основными
	технологическими уровнями являются программное и аппаратное обеспечение, уровень коммуникации
	между устройства и платформа, которая их объединяет. Среди большого разнообразия используемых
	протоколов были выбраны три основных решения: ZigBee, Z-Wave, Wi-Fi $-$ и проведён сравнительный анализ
	их технических характеристик.
	
	Вторая глава продолжает описание и сравнение выбранных протоколов уже с точки зрения безопасности.
	Были рассмотрены алгоритмы выработки и распределения ключей, шифрования и контроля целостности
	данных, используемые в протоколах.
	
	Третья часть рассказывает об угрозах, свойственных сетям на основе ZigBee, Z-Wave и Wi-Fi, а также
	содержит описание успешно проведённых атак, с целью выявления уязвимостей и дальнейшего их
	устранения.
	
	В качестве направлений для дальнейшего исследования можно выделить следующие:
	\begin{itemize}
		\item Построить матрицу угроз, которая отображает потенциальную уязвимость IoT протоколов 
		к определённым типам криптографических атак;
		\item Исходя из результатов сравнительного анализа свойств IoT протоколов и их стойкости 
		к криптографическим атакам выбрать один из протоколов для дальнейшей модификации;
		\item Обратить внимание на концепцию легковесной криптографии; предложить облегчённую 
		модификацию sponge-функции из белорусского 
		криптографического стандарта СТБ 34.101.77 (bash);
		\item Для выбранного протокола разработать модификацию, использующую исключительно 
		модифицированные белорусские криптографические алгоритмы; провести оценку предложенного 
		уровня стойкости;
		\item Реализовать прототип, демонстрирующий работу выбранного протокола с модификациями;
		\item Продемонстрировать стойкость разработанного протокола к классическим атакам на IoT решения.
	\end{itemize}
	