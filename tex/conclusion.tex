\chapter*{Заключение}
	\addcontentsline{toc}{chapter}{Заключение}
	
	Интернет вещей всё глубже проникает в жизнь конечного потребителя, набирая всё большую популярность.
	В связи с этим вопросы обеспечения безопасности конечных устройств и криптографической защиты данных
	пользователей остаются открытыми и актуальными.
	
	В первой главе текущего исследования были описаны ключевые технологии, применимые в IoT. Основными
	технологическими уровнями являются программное и аппаратное обеспечение, уровень коммуникации
	между устройства и платформа, которая их объединяет. Среди большого разнообразия используемых
	протоколов были выбраны три основных решения: ZigBee, Z-Wave, Wi-Fi $-$ и проведён сравнительный анализ
	их технических характеристик.
	
	Вторая глава продолжает описание и сравнение выбранных протоколов уже с точки зрения безопасности.
	Были рассмотрены алгоритмы выработки и распределения ключей, шифрования и контроля целостности
	данных, используемые в протоколах.
	
	Третья часть рассказывает об угрозах, свойственных сетям на основе ZigBee, Z-Wave и Wi-Fi, а также
	содержит описание успешно проведённых атак, с целью выявления уязвимостей и дальнейшего их
	устранения. В соответствии с поставленными во введении целями была построена матрицы угроз,
	которая отображает потенциальную уязвимость IoT протоколов к определённым типам криптографических 
	атак. В качестве набора базовых угроз были выбраны следующие: атака <<человек посередине>>,
	атака повторного воспроизведения, защита от <<чтения назад>>, атака понижения версии.
	
	В заключительной главе приведено описание процесса разработки прототипа, выбора технологий,
	обзор белорусского криптографического стандарта СТБ 34.101.77, мотивация использования
	и реализации аутентифицированного шифрования из этого стандарта. Также были детально описаны
	ключевые технические особенности разработанного решения, такие как установка соединения,
	выработка и распределение ключей шифрования, реализация счётчика отправленных и полученных
	сообщений.  Код прошивки для умного устройства, а также ключевых компонентов приложения
	для управляющего устройства приведен в приложениях к данной работе.
	