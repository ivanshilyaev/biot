\chapter*{Заключение}
	\addcontentsline{toc}{chapter}{Заключение}
	
	В работе рассмотрены теоритические и практические аспекты обеспечения криптографической защиты 
	данных при использовании протоколов из сферы интернета вещей.
	
	Основные результаты работы:
	
	\begin{itemize}
		\item Описаны ключевые технологии и технологические уровни, используемые в сфере интернета вещей.
		\item Выбраны три основных протокола: ZigBee, Z-Wave, Wi-Fi $-$ и проведён сравнительный анализ
		их технических характеристик.
		\item Рассмотрены алгоритмы выработки и распределения ключей шифрования и контроля целостности
		данных, используемые в выбранных протоколах, и проведён сравнительный анализ безопасности этих
		протоколов.
		\item Описаны известные угрозы и успешно проведённые атаки на различные версии протоколов.
		\item Построена матрицы угроз, отображающая потенциальную уязвимость IoT протоколов к 
		определённым типам криптографических атак. В качестве набора базовых угроз были выбраны 
		следующие: атака <<человек посередине>>, атака повторного воспроизведения, <<чтение назад>>, 
		атака понижения версии.
		\item Реализованы алгоритмы аутентифицированного шифрования и хэширования из белорусского 
		криптографического стандарта СТБ 34.101.77 на двух языках программирования: Java и C++.
		\item Разработана прошивка для умного устройства на языке программирования C++ с применением 
		упомянутого шифрования, а также прототип умной лампочки, работающий на этой прошивке. Описана 
		мотивация использования алгоритмов из выбранного стандарта.
		\item Разработано веб-приложение для управляющего устройства на языке программирования Java.
		\item Детально рассмотрены ключевые технические особенности разработанного решения, такие как 
		установка соединения, выработка и распределение ключей шифрования, реализация счётчика отправленных 
		и полученных сообщений. Код прошивки для умного устройства, а также ключевых компонентов приложения
		для управляющего устройства приведен в приложениях к данной работе.
		\item Проведена оценка скорости обмена сообщениями в протоколе и демонстрация корректной работы 
		прототипа.
	\end{itemize}

	Результаты работы были представлены на 79-ой научной конференции студентов и аспирантов БГУ.
	По итогам этой конференции к публикации готовится соответствующая статья.
