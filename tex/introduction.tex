\chapter*{Введение}
 \addcontentsline{toc}{chapter}{Введение}
 
 	Термин <<Интернет вещей>> (<<Internet of Things>>) появился более 20 лет назад, а история развития
 	технологии насчитывает почти два столетия. Среди множества определений термина можно выделить
 	следующее: интернет вещей --- это глобальная сеть объектов, подключённых к интернету, которые 
 	взаимодействуют между собой и обмениваются данными без вмешательства человека.
 	
 	Основными компонентами IoT систем являются:
 	\begin{itemize}
 		\item объекты, или <<вещи>>;
 		\item данные, которыми они обмениваются;
 		\item инфраструктура, с помощью которой осуществляется взаимодействие.
 	\end{itemize}
 	К последнему пункту можно отнести разнообразные виды соединения и каналы связи, программные 
 	средства и протоколы. Инфраструктура и её криптографический аспект представляют собой наибольший
 	практический интерес и составляют предметную область данной работы.
 	
 	Говоря о практическом применении Интернета вещей, многие отрасли выигрывают при использовании
 	этой технологии. И в каждой из этих отраслей необходимо думать о безопасности и защите данных.
 	В связи с этим возникают задачи актуализации знаний об алгоритмах и протоколах, применяемых в
 	данной сфере, их сравнении и реализации в рамках программного обеспечения, а также рассмотрения
 	вариантов модификации и улучшения этих протоколов с применением белорусской криптографии. Эти
 	задачи и легли в основу данной работы. В соответствии с задачами были поставлены следующие цели:
 	
 	\begin{enumerate}
 		\item Изучить сетевые протоколы, применяемые в сфере IoT, и провести их сравнительный анализ;
 		\item Разобрать криптографический аспект описанных в первой главе сетевых протоколов в контексте
 		используемых в них криптографических протоколов и алгоритмов;
 		\item Описать уязвимости и угрозы используемых решений;
 		\item Проанализировать возможность модификации криптографических протоколов и алгоритмов с
 		внедрением элементов белорусской криптографии.
 	\end{enumerate}
 	
 	Данная работа состоит из 4 глав, в которых последовательно раскрываются все перечисленные выше
 	вопросы.