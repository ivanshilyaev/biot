\chapter*{Введение}
 \addcontentsline{toc}{chapter}{Введение}
 
 	Термин <<Интернет вещей>> (<<Internet of Things>>) появился более 20 лет назад, а история развития
 	технологии насчитывает почти два столетия. Среди множества определений термина можно выделить
 	следующее. Интернет вещей --- это глобальная сеть объектов, подключённых к интернету, которые 
 	взаимодействуют между собой и обмениваются данными без вмешательства человека.
 	
 	Основными компонентами IoT систем являются объекты, или <<вещи>>, данные, которыми они обмениваются,
 	и инфраструктура, с помощью которой осуществляется взаимодействие. К последнему пункту можно
 	отнести разнообразные виды соединения и каналы связи, программные средства и протоколы. Именно
 	этот пункт будет интересовать нас больше всего в рамках данной работы.
 	
 	Говоря о практическом применении Интернета вещей, многие отрасли выигрывают при использовании
 	этой технологии. И в каждой из этих отраслей необходимо думать о безопасности и защите данных.
 	В связи с этим возникают задачи актуализации знаний об алгоритмах и протоколах, применяемых в
 	данной сфере, их сравнении и реализации в рамках программного обеспечения, а также рассмотрения
 	вариантов модификации и улучшения этих протоколов с применением белорусской криптографии. Эти
 	задачи и легли в основу данной работы. В соответствии с задачами были поставлены следующие цели:
 	
 	\begin{enumerate}
 		\item ...
 		\item ...
 		\item ...
 		\item ...
 	\end{enumerate}
 	
 	Данная работа состоит из ... глав, в которых последовательно раскрываются все 
 	перечисленные выше вопросы.