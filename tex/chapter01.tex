\chapter{Анализ литературы}


	\setcounter{subsection}{-1}
	\section{Технологии Интернета вещей}
	IoT включает в себя бесчисленное количество технологий и решений, и чтобы понять их все, необходимо
	потратить немало времени. Однако в целях упрощения существует возможность разбить весь IoT стек на
	четыре базовых технологических уровня, которые позволяют функционировать всему Интернету вещей.
	
	\textbf{Аппаратное обеспечение} устройств является первым из этих уровней. Устройства $-$ это те самые
	<<вещи>> в аббревиатуре IoT. Выступая в роли интерфейса между реальным и цифровым миром, они
	могут принимать разные формы и размеры, а также иметь разные уровни технологической оснащённости в
	зависимости от выполняемой задачи. Практически любой предмет может быть подключен к Интернету и
	оснащён необходимым инструментарием (сенсорами, датчиками и т.д.) в целях измерения и сбора данных.
	Единственным существенным ограничением может быть реальный практический сценарий использования.
	
	\textbf{Программное обеспечение} является элементом, который делает девайсы по-настоящему <<умными>>.
	Программы ответственны за коммуникацию с облаком, сбор данных, взаимодействие между устройствами,
	а также анализ данных в реальном времени. Более того, программное обеспечение помогает взаимодействовать
	с IoT системами на уровне приложения конечному пользователю, визуализирую обработанные данные для него.
	
	\textbf{Уровень коммуникации (или сообщения)} тесно связан с программным и аппаратным обеспечением, однако
	необходимость рассматривать его отдельно является ключевой. Этот уровень содержит средства для обмена
	информацией между умными устройствами и основным IoT миром. Он включает в себя как физическое соединение,
	так и специальные протоколы, на которых будет сделать акцент в данной работе. Выбор правильного решения
	для обмена сообщениями является ключевым при построении каждой системы. Технологии отличаются в
	зависимости от способа передачи данных и управления устройствами.
	
	Благодаря программному и аппаратному обеспечению девайсы могут считывать, что происходит вокруг, и
	коммуницировать с пользователями по специальным каналам связи. \textbf{IoT платформа} $-$ это место, в котором
	все собранные данные обрабатываются, анализируются и представляются пользователю в удобном виде.
	Её достоинством является извлечение полезных данных из большого объёма информации, который передаётся
	от устройств по каналам связи.
	
	
	\section{Используемые протоколы}
	
	Существует множество разнообразных способов взаимодействия умных устройств между собой. Поэтому
	при выборе протоколов для Интернета вещей часто возникает вопрос: есть ли реальная необходимость
	разработки новых решений, когда хорошо зарекомендовавшие себя протоколы сети Интернет уже
	используются повсеместно десятилетиями? Причина для этого кроется в том, что существующие протоколы
	часто оказываются недостаточно эффективными и слишком энергоёмкими для работы с возникающими
	IoT технологиями. Поэтому речь пойдёт об альтернативных решениях, посвящённых именно IoT системам.
	
	Одна из возможных классификаций разбивает все протоколы на три группы: близкого, среднего и дальнего
	действия. Наиболее ярким представителем первой группы является Bluetooth, который несмотря на свою
	повсеместную распространённость остаётся далеко не лучшим решением, особенно при передаче больших
	объёмов данных. К последней группе относят такие протоколы как NB-IoT, LTE Cat-M1, LoRa WAN и SigFox.
	Эти решения являются весьма современными и продвинутыми, однако используются часто в масштабах
	предприятий. Наша же цель заключается в изучении решений, применимых к простым пользователям 
	IoT систем, поэтому данный раздел будет преимущественно сконцентрирован вокруг второй группы, 
	а именно протоколов средней зоны действия.
	
	% вставить картинку со сравнением протоколом дальнего действия
	

	\subsection{ZigBee}
	Этот популярный стандарт беспроводных сетей находит свое наиболее частое применение в системах 
	управления дорожным движением, бытовой электронике и машиностроении. Созданный на базе стандарта
	IEEE 802.15.4, Zigbee поддерживает низкое энергопотребление, безопасность и надежность.
	
	
	\subsection{Z-Wave}
	
	
	\subsection{Wi-Fi}
	Построенный на базе стандарта IEEE 802.11, Wi-Fi остаётся самым распространённым и наиболее
	известным беспроводным протоколом взаимодействия. Его широкое использование в мире IoT в
	основном ограничено энергопотреблением выше среднего по причине удержание качественного сигнала
	и быстрой передачи данных для лучшего соединения и надёжности. Несмотря на это Wi-Fi является
	ключевой технологией в развитии и распространении IoT.
	
	