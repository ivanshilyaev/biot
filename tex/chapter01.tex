\chapter{Анализ литературы}


	\setcounter{subsection}{-1}
	\section{Используемые протоколы}
	
	Существует множество разнообразных способов взаимодействия умных устройств между собой. Поэтому
	при выборе протоколов для Интернета вещей часто возникает вопрос: есть ли реальная необходимость
	разработки новых решений, когда хорошо зарекомендовавшие себя протоколы сети Интернет уже
	используются повсеместно десятилетиями? Причина для этого кроется в том, что существующие протоколы
	часто оказываются недостаточно эффективными и слишком энергоёмкими для работы с возникающими
	IoT технологиями. Поэтому речь пойдёт об альтернативных решениях, посвящённых именно IoT системам.
	
	Одна из возможных классификаций разбивает все протоколы на три группы: близкого, среднего и дальнего
	действия. Наиболее ярким представителем первой группы является Bluetooth, который несмотря на свою
	повсеместную распространённость остаётся далеко не лучшим решением, особенно при передаче больших
	объёмов данных. К последней группе относят такие протоколы как NB-IoT, LTE Cat-M1, LoRa WAN и SigFox.
	Эти решения являются весьма современными и продвинутыми, однако используются часто в масштабах
	предприятий. Наша же цель заключается в изучении решений, применимых к простым пользователям 
	IoT систем, поэтому данный раздел будет преимущественно сконцентрирован вокруг второй группы, 
	а именно протоколов средней зоны действия.
	
	% вставить картинку со сравнением протоколом дальнего действия
	
	
	\subsection{ZigBee}
	
	
	\subsection{Z-Wave}
	
	
	\subsection{Wi-Fi}