\chapter{Анализ литературы}


	\setcounter{subsection}{-1}
	\section{Технологии Интернета вещей}
	IoT включает в себя бесчисленное количество технологий и решений, и чтобы понять их все, необходимо
	потратить немало времени. Однако в целях упрощения существует возможность разбить весь IoT стек на
	четыре базовых технологических уровня, которые позволяют функционировать всему Интернету вещей.
	
	\textbf{Аппаратное обеспечение} устройств является первым из этих уровней. Устройства $-$ это те самые
	<<вещи>> в аббревиатуре IoT. Выступая в роли интерфейса между реальным и цифровым миром, они
	могут принимать разные формы и размеры, а также иметь разные уровни технологической оснащённости в
	зависимости от выполняемой задачи. Практически любой предмет может быть подключен к Интернету и
	оснащён необходимым инструментарием (сенсорами, датчиками и т.д.) в целях измерения и сбора данных.
	Единственным существенным ограничением может быть реальный практический сценарий использования.
	
	\textbf{Программное обеспечение} является элементом, который делает девайсы по-настоящему <<умными>>.
	Программы ответственны за коммуникацию с облаком, сбор данных, взаимодействие между устройствами,
	а также анализ данных в реальном времени. Более того, программное обеспечение помогает взаимодействовать
	с IoT системами на уровне приложения конечному пользователю, визуализирую обработанные данные для него.
	
	\textbf{Уровень коммуникации (или сообщения)} тесно связан с программным и аппаратным обеспечением, однако
	необходимость рассматривать его отдельно является ключевой. Этот уровень содержит средства для обмена
	информацией между умными устройствами и основным IoT миром. Он включает в себя как физическое соединение,
	так и специальные протоколы, на которых будет сделать акцент в данной работе. Выбор правильного решения
	для обмена сообщениями является ключевым при построении каждой системы. Технологии отличаются в
	зависимости от способа передачи данных и управления устройствами.
	
	Благодаря программному и аппаратному обеспечению девайсы могут считывать, что происходит вокруг, и
	коммуницировать с пользователями по специальным каналам связи. \textbf{IoT платформа} $-$ это место, в котором
	все собранные данные обрабатываются, анализируются и представляются пользователю в удобном виде.
	Её достоинством является извлечение полезных данных из большого объёма информации, который передаётся
	от устройств по каналам связи.
	
	
	\section{Используемые протоколы}
	
	Существует множество разнообразных способов взаимодействия умных устройств между собой. Поэтому
	при выборе протоколов для Интернета вещей часто возникает вопрос о том, есть ли реальная необходимость
	разработки новых решений, в то время как хорошо зарекомендовавшие себя протоколы сети Интернет уже
	используются повсеместно десятилетиями. Причина для этого кроется в том, что существующие протоколы
	часто оказываются недостаточно эффективными и слишком энергоёмкими для работы с возникающими
	IoT технологиями. Поэтому речь пойдёт об альтернативных решениях, посвящённых именно IoT системам.
	
	Одна из возможных классификаций разбивает все протоколы на три группы: близкого, среднего и дальнего
	действия. Наиболее ярким представителем первой группы является Bluetooth, который несмотря на свою
	повсеместную распространённость остаётся далеко не лучшим решением, особенно при передаче больших
	объёмов данных. К последней группе относят такие протоколы как NB-IoT, LTE Cat-M1, LoRa WAN и SigFox.
	Эти решения являются весьма современными и продвинутыми, однако используются часто в масштабах
	предприятий. Наша же цель заключается в изучении решений, применимых к простым пользователям 
	IoT систем, поэтому данный раздел будет преимущественно сконцентрирован вокруг второй группы, 
	а именно протоколов средней зоны действия.
	
	% более подробно описать Bluetooth
	
	% вставить картинку со сравнением протоколом дальнего действия
	

	\subsection{ZigBee}
	Этот популярный стандарт беспроводных сетей находит свое наиболее частое применение в системах 
	управления дорожным движением, бытовой электронике и машиностроении. Созданный на базе стандарта
	IEEE 802.15.4, Zigbee поддерживает высокую отказоустойчивость, низкое энергопотребление, безопасность
	и надежность.
	
	Протокол ZigBee описывает беспроводные персональные сети (Wireless personal area network, WPAN).
	Технология, определённая спецификацией Zig- \newline -Bee, подразумевает более дешёвое производство по
	сравнению с другими беспроводными персональными сетями, такими как Bluetooth, или более общими
	технологиями, такими как Wi-Fi. ZigBee обычно используется в решениях, требующих долгого времени
	работы (например, от батареи), и безопасной передачи данных.
	Индивидуальные устройства в подобной сети могут работать на одной батарее до двух лет.
	Сети на основе ZigBee характеризуются довольно низкой пропускной способностью (до 250 Кбит/с) и
	дальностью связи между узлами до 100 метров (на открытой местности это значение может достигать
	200 метров). Протокол был задуман в 1998 году. Первоначальная спецификация была признана стандартом 
	IEEE в 2003 году, а первые модули, совместимые с ZigBee, появились в массовой продаже в начале 2006 года.
	
	Существует три класса устройств Zigbee:
	
	\begin{enumerate}
		\item Координатор Zigbee (ZC). Он образует корень сетевого дерева и может соединяться с другими 
		сетями, являясь самым функциональным устройством. В каждой сети есть только один координатор 
		Zigbee, поскольку именно это устройство является создателем сети. Однако спецификация Zigbee 
		LightLink позволяет работать без координатора, что делает её более пригодной для использования 
		в готовых домашних продуктах. Координатор хранит информацию о сети, выполняя в том числе функции 
		удостоверяющего центра и хранилища ключей безопасности.
		\item Маршрутизатор Zigbee (ZR). Помимо выполнения функции приложения, маршрутизатор может 
		выступать в качестве промежуточного звена, передавая данные от других устройств.
		\item Конечное устройство Zigbee (ZED). Содержит достаточно функций, чтобы общаться с координатором 
		или маршрутизатором и не может передавать данные от других устройств. Такое взаимодействие 
		позволяет узлу находиться в спящем состоянии значительную часть времени, что обеспечивает 
		длительное время автономной работы. ZED требует наименьшего объема памяти и поэтому может 
		быть дешевле в производстве, чем координатор или маршрутизатор.
	\end{enumerate}

	Посмотрим на классы устройств ZigBee на примере беспроводного выключателя света. Узел Zigbee на лампе 
	способен постоянно принимать сигнал, так как он подключён к электрической сети. В то же время выключатель, 
	работающий от батарейки, будет находиться в спящем режиме большую часть времени: до тех пор, пока 
	его состояние не будет изменено. В этом случае выключатель просыпается, посылает команду лампе, 
	дожидается подтверждения и возвращается в спящий режим. В подобной сети узел лампы должен быть 
	по меньшей мере маршрутизатором сети, узел выключателя обычно является конечным устройством.
	
	ZigBee был разработан как стандарт для радиосетей с ячеистой (mesh) топологией, предназначенных
	для использования в системах телеметрии, для связи между различными типами датчиков, устройств
	мониторинга, а также для беспроводного считывания результатов измерений с приборов учета энергии,
	тепла и т.д. Кроме того, ZigBee поддерживает сети с топологией <<звезда>> и <<дерево>>. В каждой 
	сети должно быть одно устройство-координатор. В сетях с топологией <<звезда>> координатор должен 
	быть центральным узлом. Как древовидные, так и ячеистые сети позволяют использовать маршрутизаторы 
	Zigbee для расширения связи на сетевом уровне.
	Стандарт ZigBee представляет собой относительно простой, устойчивый к ошибкам связи и
	несанкционированному считыванию, пакетный протокол обмена данными, который часто реализуется в
	устройствах с относительно небольшими требованиями, таких как микроконтроллеры, датчики и т.д.
	
	ZigBee легко устанавливать и обслуживать, поскольку он основан на самосборке и самовосстанавливающейся
	топологии сети. Он также легко масштабируется до тысяч узлов, а максимальное число узлов в подобной сети
	может достигать 65000. В настоящее время существует множество поставщиков, предлагающих устройства,
	поддерживающие этот открытый стандарт.
	
	Устройства, использующие ZigBee, преимущественно включают в себя беспроводные лампочки 
	и выключатели света, системы управления дорожным движением и другое потребительское и промышленное 
	оборудование. Типичными сферами применения являются:
	\begin{itemize}
		\item домашняя автоматизация;
		\item промышленные системы управления;
		\item сбор медицинских данных;
		\item оповещение о задымлении и несанкционированном проникновении;
		\item автоматизация зданий.
	\end{itemize}

	Zigbee Alliance $-$ это группа компаний, которые поддерживают и публикуют стандарт Zigbee. Название 
	Zigbee является зарегистрированной торговой маркой этой группы и представляет из себя не просто 
	технический стандарт. Организация публикует материалы, которые позволяют производителям 
	создавать совместимые продукты. Связь между IEEE 802.15.4 и Zigbee похожа на связь между 
	IEEE 802.11 и Wi-Fi Alliance.
	
	За годы существования альянса его членами стали более 500 компаний, включая Comcast, Ikea, Legrand, 
	Samsung SmartThings и Amazon. Zigbee Alliance имеет три уровня членства. Члены первой группы 
	имеют доступ к готовым спецификациям и стандартам Zigbee, а члены второй $-$ право голоса, 
	играя роль в развитии Zigbee и имея ранний доступ к спецификациям и стандартам для разработки 
	продуктов.
	
	
	\subsection{Z-Wave}
	
	
	\subsection{Wi-Fi}
	Построенный на базе стандарта IEEE 802.11, Wi-Fi остаётся самым распространённым и наиболее
	известным беспроводным протоколом взаимодействия. Его широкое использование в мире IoT в
	основном ограничено энергопотреблением выше среднего по причине удержание качественного сигнала
	и быстрой передачи данных для лучшего соединения и надёжности. Несмотря на это Wi-Fi является
	ключевой технологией в развитии и распространении IoT.
	
	% картинка со сравнением ZigBee, Z-Wave, Wi-Fi
	
	