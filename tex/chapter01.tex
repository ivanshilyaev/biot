\chapter{Анализ литературы}


	\setcounter{subsection}{-1}
	\section{Технологии Интернета вещей}
	IoT включает в себя бесчисленное количество технологий и решений, и чтобы понять их все, необходимо
	потратить немало времени. Однако в целях упрощения существует возможность разбить весь IoT стек на
	четыре базовых технологических уровня, которые позволяют функционировать всему Интернету вещей.
	
	\textbf{Аппаратное обеспечение} устройств является первым из этих уровней. Устройства $-$ это те самые
	<<вещи>> в аббревиатуре IoT. Выступая в роли интерфейса между реальным и цифровым миром, они
	могут принимать разные формы и размеры, а также иметь разные уровни технологической оснащённости в
	зависимости от выполняемой задачи. Практически любой предмет может быть подключен к Интернету и
	оснащён необходимым инструментарием (сенсорами, датчиками и т.д.) в целях измерения и сбора данных.
	Единственным существенным ограничением может быть реальный практический сценарий использования.
	
	\textbf{Программное обеспечение} является элементом, который делает девайсы по-настоящему <<умными>>.
	Программы ответственны за коммуникацию с облаком, сбор данных, взаимодействие между устройствами,
	а также анализ данных в реальном времени. Более того, программное обеспечение помогает взаимодействовать
	с IoT системами на уровне приложения конечному пользователю, визуализирую обработанные данные для него.
	
	\textbf{Уровень коммуникации (или сообщения)} тесно связан с программным и аппаратным обеспечением, однако
	необходимость рассматривать его отдельно является ключевой. Этот уровень содержит средства для обмена
	информацией между умными устройствами и основным IoT миром. Он включает в себя как физическое соединение,
	так и специальные протоколы, на которых будет сделать акцент в данной работе. Выбор правильного решения
	для обмена сообщениями является ключевым при построении каждой системы. Технологии отличаются в
	зависимости от способа передачи данных и управления устройствами.
	
	Благодаря программному и аппаратному обеспечению девайсы могут считывать, что происходит вокруг, и
	коммуницировать с пользователями по специальным каналам связи. \textbf{IoT платформа} $-$ это место, в котором
	все собранные данные обрабатываются, анализируются и представляются пользователю в удобном виде.
	Её достоинством является извлечение полезных данных из большого объёма информации, который передаётся
	от устройств по каналам связи.
	
	
	\section{Используемые протоколы}
	
	Существует множество разнообразных способов взаимодействия умных устройств между собой. Поэтому
	при выборе протоколов для Интернета вещей часто возникает вопрос о том, есть ли реальная необходимость
	разработки новых решений, в то время как хорошо зарекомендовавшие себя протоколы сети Интернет уже
	используются повсеместно десятилетиями. Причина для этого кроется в том, что существующие протоколы
	часто оказываются недостаточно эффективными и слишком энергоёмкими для работы с возникающими
	IoT технологиями. Поэтому речь пойдёт об альтернативных решениях, посвящённых именно IoT системам.
	
	Одна из возможных классификаций разбивает все протоколы на три группы: близкого, среднего и дальнего
	действия. Наиболее ярким представителем первой группы является Bluetooth, который несмотря на свою
	повсеместную распространённость остаётся далеко не лучшим решением, особенно при передаче больших
	объёмов данных. К последней группе относят такие протоколы как NB-IoT, LTE Cat-M1, LoRa WAN и SigFox.
	Эти решения являются весьма современными и продвинутыми, однако используются часто в масштабах
	предприятий. Наша же цель заключается в изучении решений, применимых к простым пользователям 
	IoT систем, поэтому данный раздел будет преимущественно сконцентрирован вокруг второй группы, 
	а именно протоколов средней зоны действия.
	
	% более подробно описать Bluetooth
	
	% вставить картинку со сравнением протоколом дальнего действия
	

	\subsection{ZigBee}
	Этот популярный стандарт беспроводных сетей находит свое наиболее частое применение в системах 
	управления дорожным движением, бытовой электронике и машиностроении. Созданный на базе стандарта
	IEEE 802.15.4, Zigbee поддерживает высокую отказоустойчивость, низкое энергопотребление, безопасность
	и надежность.
	
	Протокол ZigBee описывает беспроводные персональные сети (Wireless personal area network, WPAN).
	Технология, определённая спецификацией Zig- \newline -Bee, подразумевает более дешёвое производство по
	сравнению с другими беспроводными персональными сетями, такими как Bluetooth, или более общими
	технологиями, такими как Wi-Fi. ZigBee обычно используется в решениях, требующих долгого времени
	работы (например, от батареи), и безопасной передачи данных.
	Индивидуальные устройства в подобной сети могут работать на одной батарее до двух лет.
	Сети на основе ZigBee характеризуются довольно низкой пропускной способностью (до 250 Кбит/с) и
	дальностью связи между узлами до 100 метров (на открытой местности это значение может достигать
	200 метров). Протокол был задуман в 1998 году. Первоначальная спецификация была признана стандартом 
	IEEE в 2003 году, а первые модули, совместимые с ZigBee, появились в массовой продаже в начале 2006 года
	 \cite{zigbee-certified-products}.
	
	Существует три класса устройств Zigbee:
	
	\begin{enumerate}
		\item Координатор Zigbee (ZC). Он образует корень сетевого дерева и может соединяться с другими 
		сетями, являясь самым функциональным устройством. В каждой сети есть только один координатор 
		Zigbee, поскольку именно это устройство является создателем сети. Однако спецификация Zigbee 
		LightLink позволяет работать без координатора, что делает её более пригодной для использования 
		в готовых домашних продуктах. Координатор хранит информацию о сети, выполняя в том числе функции 
		удостоверяющего центра и хранилища ключей безопасности.
		\item Маршрутизатор Zigbee (ZR). Помимо выполнения функции приложения, маршрутизатор может 
		выступать в качестве промежуточного звена, передавая данные от других устройств.
		\item Конечное устройство Zigbee (ZED). Содержит достаточно функций, чтобы общаться с координатором 
		или маршрутизатором и не может передавать данные от других устройств. Такое взаимодействие 
		позволяет узлу находиться в спящем состоянии значительную часть времени, что обеспечивает 
		длительное время автономной работы. ZED требует наименьшего объема памяти и поэтому может 
		быть дешевле в производстве, чем координатор или маршрутизатор.
	\end{enumerate}

	Посмотрим на классы устройств ZigBee на примере беспроводного выключателя света. Узел Zigbee на лампе 
	способен постоянно принимать сигнал, так как он подключён к электрической сети. В то же время выключатель, 
	работающий от батарейки, будет находиться в спящем режиме большую часть времени: до тех пор, пока 
	его состояние не будет изменено. В этом случае выключатель просыпается, посылает команду лампе, 
	дожидается подтверждения и возвращается в спящий режим. В подобной сети узел лампы должен быть 
	по меньшей мере маршрутизатором сети, узел выключателя обычно является конечным устройством.
	
	ZigBee был разработан как стандарт для радиосетей с ячеистой (mesh) топологией, предназначенных
	для использования в системах телеметрии, для связи между различными типами датчиков, устройств
	мониторинга, а также для беспроводного считывания результатов измерений с приборов учета энергии,
	тепла и т.д. Кроме того, ZigBee поддерживает сети с топологией <<звезда>> и <<дерево>>. В каждой 
	сети должно быть одно устройство-координатор. В сетях с топологией <<звезда>> координатор должен 
	быть центральным узлом. Как древовидные, так и ячеистые сети позволяют использовать маршрутизаторы 
	Zigbee для расширения связи на сетевом уровне.
	Стандарт ZigBee представляет собой относительно простой, устойчивый к ошибкам связи и
	несанкционированному считыванию, пакетный протокол обмена данными, который часто реализуется в
	устройствах с относительно небольшими требованиями, таких как микроконтроллеры, датчики и т.д.
	
	ZigBee легко устанавливать и обслуживать, поскольку он основан на самосборке и самовосстанавливающейся
	топологии сети. Он также легко масштабируется до тысяч узлов, а максимальное число узлов в подобной сети
	может достигать 65000. В настоящее время существует множество поставщиков, предлагающих устройства,
	поддерживающие этот открытый стандарт.
	
	Устройства, использующие ZigBee, преимущественно включают в себя беспроводные лампочки 
	и выключатели света, системы управления дорожным движением и другое потребительское и промышленное 
	оборудование. Типичными сферами применения являются:
	\begin{itemize}
		\item домашняя автоматизация;
		\item промышленные системы управления;
		\item сбор медицинских данных;
		\item оповещение о задымлении и несанкционированном проникновении;
		\item автоматизация зданий.
	\end{itemize}

	Zigbee Alliance $-$ это группа компаний, которые поддерживают и публикуют стандарт Zigbee
	 \cite{zigbee-alliance}. Название 
	Zigbee является зарегистрированной торговой маркой этой группы и представляет из себя не просто 
	технический стандарт. Организация публикует материалы, которые позволяют производителям 
	создавать совместимые продукты. Связь между IEEE 802.15.4 и Zigbee похожа на связь между 
	IEEE 802.11 и Wi-Fi Alliance.
	
	За годы существования альянса его членами стали более 500 компаний, включая Comcast, Ikea, Legrand, 
	Samsung SmartThings и Amazon. Zigbee Alliance имеет три уровня членства. Члены первой группы 
	имеют доступ к готовым спецификациям и стандартам Zigbee, а члены второй $-$ право голоса, 
	играя роль в развитии Zigbee и имея ранний доступ к спецификациям и стандартам для разработки 
	продуктов.
	
	
	\subsection{Z-Wave}
	Z-Wave $-$ это протокол беспроводной связи, используемый в основном для домашней автоматизации. 
	Он применяется преимущественно для управления бытовой техникой и другими устройствами, такими 
	как освещение, охранные системы, термостаты, окна, замки, бассейны и открыватели гаражных дверей.
	Как и другие протоколы, предназначенные для рынка автоматизации дома и офиса, система Z-Wave может 
	управляться через Интернет со смартфона, планшета или компьютера, а также локально через умную 
	колонку или хаб, настенную панель со шлюзом Z-Wave или центральным устройством управления. 
	Z-Wave обеспечивает совместимость на прикладном уровне между системами управления домом различных 
	производителей, входящих в её альянс. Число совместимых продуктов Z-Wave значительно растёт, 
	к 2019 году их количество составляло более 2600 \cite{z-wave-certified-products}.
	
	Протокол Z-Wave был разработан датской компанией Zensys, расположенной в Копенгагене, 
	в 1999 году. В этом же году была представлена потребительская 
	систему управления светом. Набор микросхем серии 100 был выпущен в 2003 году, а серии 200 $-$ 
	в мае 2005 года. Микросхема серии 500, также известная как Z-Wave Plus, была выпущена в марте 2013 года,
	с увеличенным в четыре раза объемом памяти, улучшенным радиусом действия беспроводной связи и 
	увеличенным временем автономной работы. Технология начала распространяться в Северной Америке 
	примерно в 2005 году, когда пять компаний приняли Z-Wave и сформировали Z-Wave Alliance, целью 
	которого является продвижение использования технологии Z-Wave \cite{z-wave-alliance}. При этом 
	все продукты компаний, входящих в альянс, должны быть совместимы. В том же 2005 году технология
	получила первые инвестиции.
	
	В настоящее время Z-Wave Alliance насчитывает более 700 производителей. Основными членами альянса 
	являются ADT Corporation, Assa Abloy, Jasco, Leedarson, LG Uplus, Nortek Security \& Control, Ring, Silicon Labs, 
	SmartThings, Trane Technologies и Vivint.
	
	Взаимодействие Z-Wave на уровне приложений обеспечивает обмен информацией между устройствами 
	и позволяет всем аппаратным и программным средствам Z-Wave работать вместе. Технология беспроводной 
	ячеистой сети (аналогичная с ZigBee) позволяет любому узлу напрямую или косвенно общаться с соседними 
	узлами, управляя любыми дополнительными узлами. Узлы, находящиеся в радиусе действия, общаются друг 
	с другом напрямую. Если они не находятся в радиусе действия, они могут связаться с другим узлом, 
	который расположен в зоне действия обоих узлов, чтобы получить доступ и обменяться информацией.
	
	Z-Wave разработан для обеспечения надёжной передачи небольших пакетов данных с низкой задержкой 
	на скорости до 100 кбит/с. Пропускная способность составляет 40 кбит/с и подходит для приложений 
	управления и датчиков, в отличие от Wi-Fi и других систем беспроводных локальных сетей на базе IEEE 802.11, 
	которые предназначены в основном для высокой скорости передачи данных. Расстояние связи между 
	двумя узлами составляет около 40 метров.
	
	Z-Wave функционирует в диапазоне частот до 1 ГГц. Этот диапазон конкурирует с некоторыми беспроводными 
	телефонами и другими устройствами бытовой электроники, но позволяет избежать помех в виде Wi-Fi, Bluetooth 
	и других систем, работающих в переполненном диапазоне 2,4 ГГц.
	
	Z-Wave использует архитектуру ячеистой сети с маршрутизацией от источника. Устройства могут 
	связываться друг с другом, используя промежуточные узлы для активной маршрутизации и обхода 
	бытовых препятствий или мертвых зон. Таким образом, сеть Z-Wave может охватывать гораздо большее 
	расстояние, чем радиус действия одного узла. Однако при наличии нескольких таких переходов может 
	возникнуть небольшая задержка между управляющей командой и желаемым результатом.
	
	Простейшая сеть представляет собой одно управляемое устройство и первичный контроллер. Дополнительные 
	устройства могут быть добавлены в любое время, как и вторичные контроллеры,включая приложения 
	для смартфонов и ПК, разработанные для управления и контроля сети Z-Wave. Сеть может включать 
	до 232 устройств, а при необходимости увеличения количества устройств возможно объединение сетей.
	
	Каждой сети Z-Wave назначается идентификатор, а каждое устройство содержит идентификатор узла. 
	Идентификатор сети (Home ID) $-$ это общая идентификация всех узлов, принадлежащих к одной логической 
	сети Z-Wave. Сетевой ID имеет длину 4 байта и присваивается каждому устройству первичным контроллером, 
	когда устройство добавляется в сеть. Узлы с разными сетевыми идентификаторами не могут взаимодействовать 
	друг с другом. Идентификатор узла $-$ это адрес одного узла в сети. Идентификатор узла имеет 
	длину 1 байт и является уникальным в своей сети.
	
	Чип Z-Wave оптимизирован для устройств, работающих от батарей, и большую часть времени находится 
	в режиме энергосбережения, чтобы потреблять меньше энергии, просыпаясь только для выполнения своей 
	функции. В ячеистых сетях Z-Wave каждое устройство в доме распространяет беспроводные сигналы по 
	всему дому, что приводит к низкому энергопотреблению, позволяя устройствам работать годами без 
	необходимости замены батарей. Чтобы устройства Z-Wave могли передавать сторонние сообщения, 
	они не должны быть в спящем режиме. Поэтому устройства, работающие от батарей, не предназначены для 
	использования в качестве ретрансляторов.
	
	
	\subsection{Wi-Fi}
	Построенный на базе стандарта IEEE 802.11, Wi-Fi остаётся самым распространённым и наиболее
	известным беспроводным протоколом взаимодействия. Его широкое использование в мире IoT в
	основном ограничено энергопотреблением выше среднего по причине удержание качественного сигнала
	и быстрой передачи данных для лучшего соединения и надёжности. Несмотря на это Wi-Fi является
	ключевой технологией в развитии и распространении IoT.
	
	Для создания сети Wi-Fi требуются устройства, способные передавать беспроводные сигналы, то есть 
	такие устройства, как телефоны, компьютеры или маршрутизаторы. В домашних условиях маршрутизатор 
	используется для передачи интернет-соединения из общественной сети в частную домашнюю или офисную 
	сеть. Wi-Fi обеспечивает подключение к Интернету близлежащих устройств, находящихся в определенном 
	радиусе действия. Другой способ использования Wi-Fi $-$ создание точки доступа.
	
	Wi-Fi использует радиоволны, которые передают информацию на определенных частотах. Двумя основными
	частотами являются 2,4 ГГц и 5 ГГц. Оба частотных диапазона имеют ряд каналов, по которым могут 
	работать различные беспроводные устройства, что помогает распределить нагрузку таким образом, 
	чтобы индивидуальные соединения устройств не прерывались. Это в значительной степени предотвращает 
	переполнение беспроводных сетей.
	
	Диапазон в 100 метров является типичным для стандартного Wi-Fi соединения. Однако чаще всего радиус 
	действия ограничивается 10-35 метрами. На эффективное покрытие влияет мощность антенны 
	и частота передачи. Дальность и скорость Wi-Fi подключения к Интернету зависит от окружающей среды 
	и от того, обеспечивает ли оно внутреннее или внешнее покрытие.
	
	Технология Wi-Fi была создана в 1998 году. В 2018 году Wi-Fi Alliance \cite{wi-fi-alliance} ввёл упрощенную 
	нумерацию поколений Wi-Fi для обозначения оборудования, поддерживающего Wi-Fi 4 (802.11n), 
	Wi-Fi 5 (802.11ac) и Wi-Fi 6 (802.11ax). Эти поколения имеют высокую степень обратной совместимости 
	с предыдущими версиями. Альянс заявил, что уровень поколения 4, 5 или 6 может быть указан в 
	пользовательском интерфейсе при подключении, наряду с уровнем сигнала.
	
	\begin{table}[h]
		\centering
		\begin{tabular}{ | l | l | l | l | l | }
			\hline
			Generation & IEEE Standard & Bands & Max data rate & Year \\ \hline
			Wi-Fi 0 & 802.11 & 2.4 GHz & 2 Mbit/s & 1997 \\ \hline
			Wi-Fi 1 & 802.11b & 2.4 GHz & 11 Mbit/s & 1999 \\ \hline
			Wi-Fi 2 & 802.11a & 5 GHz & 54 Mbit/s & 1999 \\ \hline
			Wi-Fi 3 & 802.11g & 2.4 GHz & 54 Mbit/s & 2003 \\ \hline
			Wi-Fi 4 & 802.11n & 2.4/5 GHz & 600 Mbit/s & 2008 \\ \hline
			Wi-Fi 5 & 802.11ac & 5 GHz & 6933 Mbit/s & 2014 \\ \hline
			Wi-Fi 6 & 802.11ax & 2.4/5 GHz & 9608 Mbit/s & 2019 \\ \hline
			Wi-Fi 6E & 802.11ax & 6 GHz & 9608 Mbit/s & 2020 \\
			\hline
		\end{tabular}
		\caption{Основные поколения Wi-Fi}
		\label{table 1}
	\end{table}
	
	Основные поколения Wi-Fi представлены в Таблице \ref{table 1}. Однако для технологии IoT особый интерес 
	представляют только некоторые из этих поколений, а именно:
	
	\begin{itemize}
		\item IEEE 802.11b/g/n. Эти стандарты отличаются относительно небольшим радиусом действия.
		Они функционируют в полосе частот 2400—2483,5 МГц. В стандарте  IEEE 802.11n, выпущенном
		в 2008 году, скорость соединения была существенно увеличена. Однако новая скорость может
		быть достигнута лишь в одном из трёх режимов работы, в котором не поддерживается обратная
		совместимость со стандартами IEEE 802.11b/g. Данные стандарты являются весьма популярными
		в устройствах для домашней автоматизации, где не всегда нужны огромные скорости передачи
		данных или большой радиус покрытия, а существенным параметром является энергоэффективность.
		\item IEEE 802.11ah. Этот протокол беспроводных сетей был опубликован в 2017 году под названием 
		Wi-Fi HaLow. Он использует освобожденные от лицензий полосы частот 900 МГц для обеспечения 
		сетей Wi-Fi с увеличенной дальностью действия по сравнению с обычными сетями Wi-Fi, работающими 
		в диапазонах 2,4 ГГц и 5 ГГц. Он также отличается более низким энергопотреблением, что позволяет 
		создавать большие группы станций или датчиков, которые взаимодействуют для обмена сигналами, 
		поддерживая концепцию Интернета вещей. Благодаря низкому энергопотреблению протокол конкурирует 
		с Bluetooth и имеет дополнительные преимущества в виде более высокой скорости передачи данных 
		и более широкого радиуса действия.
	\end{itemize}
	
	
	\section{Сравнение протоколов}
	
		\begin{table}[h]
		\centering
		\begin{tabular}{ | l | l | l | l | }
			\hline
			 & ZigBee & Z-Wave & Wi-Fi \\ \hline
			Standard & IEEE 802.15.4 & IEEE 802.15.4 & IEEE 802.11 \\ \hline
			Max data rate & 250 Kbit/s & 100 Kbit/s & 300+ Mbit/s \\ \hline
			Power consumption & Low & Low & High \\ \hline
			Bands & 2.4 GHz  & 908.42 MHz & 2.4 GHz/5 GHz \\ \hline
			Network topology & Mesh & Mesh & Star \\
			\hline
		\end{tabular}
		\caption{Сравнение основных протоколов IoT}
		\label{table 2}
	\end{table}

	В Таблице \ref{table 2} приведено сравнение основных характеристик трёх подробно рассмотренных
	протоколов. В сравнение не включена дальность действия: говоря о Wi-Fi, радиус непосредственного
	взаимодействия между двумя устройствами обычно является большим, однако сети на основе
	ZigBee и Z-Wave, как указано в таблице, имеют ячеистую топологию, за счёт чего они могут использовать
	промежуточные устройства для передачи сигнала и увеличения радиуса действия.
	
	Говоря о технических характеристиках, отличие ZigBee от Z-Wave невелико: Z-Wave выделяется только
	используемой частотой. Но кроме этого существует разница в распространении устройств на основе
	обоих протоколов. Z-Wave получил более широкое распространение в США, в то время как ZigBee больше
	популярен в Европе. Существует ещё одно небольшое отличие, которое заключается в частотных диапазонах. 
	В Северной Америке, 
	Европе, и ряде других стран под Z-Wave и другие протоколы отводятся разные диапазоны частот. Однако
	в большинстве случаев это не оказывает большого влияния, поскольку вне зависимости от используемой 
	частоты схожие устройства, как правило, обладают одинаковым функционалом.
	
	Немало важным фактором в сравнении является стоимость конечных устройств. В данном случае она может
	заметно варьироваться. Говоря, например, о домашней автоматизации, стоимость устройств, поддерживающих
	Wi-Fi, оказывается выше. Более дешёвая цена устройства на основе ZigBee и Z-Wave достигается
	за счёт специализированных модулей и чипов.
	
	Но кроме цены комплектующих влияние на стоимость оказывает способ управления конечными устройствами.
	Гаджеты на основе Wi-Fi могут управляться с любого смартфона, в то время как для управления ZigBee и Z-Wave
	сетями в подавляющем большинстве случаев требуется некоторое промежуточное устройство: хаб. Хаб
	позволяет преобразовывать сигналы от устройств в тот же самый Wi-Fi, добавляя возможность управления
	практически с любого устройства. Хаб имеет дополнительную стоимость. Устройства, совместимые с  Wi-Fi,
	являются более дорогими, поскольку дают возможность обходиться без него.
	
	Наконец, стоит отметить, что для каждого протокола с каждом годом появляется всё больше производителей.
	В связи с этим встаёт вопрос совместимости между устройствами различных производителей. Не редки случаи,
	когда несколько устройств не имеют возможности взаимодействия, несмотря на то, что они работают
	на одном протоколе. А при совместном использовании, например, ZigBee и Z-Wave, стоит быть ещё более
	внимательным и осторожным при выборе устройств.
