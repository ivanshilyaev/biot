\chapter{Разработка собственного решения с использованием белорусской криптографии}

	\section{Поиск существующих имплементаций}
	
	Первой задачей практической части было нахождение существующих реализаций подробно описанных
	выше протоколов с целью дальнейшей работы над криптографическим аспектом этих реализаций.
	В качестве основного протокола для модификаций был выбран протокол ZigBee. Были поставлены
	следующие задачи:
	
	\begin{itemize}
		\item поиск имплементации протокола ZigBee, позволяющей вносить модификации; 
		\item поиск устройств (микроконтроллеров), способных работать на этой имплементации;
		\item изменение или полная замена криптографической составляющей в имплементации.
	\end{itemize}

	К сожалению, открытых реализация оказалось немного. Практически не было найдено библиотек,
	реализующих в полной мере последнюю версию протокола. Проблема заключается в том, что сама
	спецификация находится в закрытом доступе. Для получения спецификации необходимо стать
	членом ZigBee Alliance, что осуществляется на платной основе. Аналогично, все имплементации
	протокола ведущими технологическими компания также являются закрытыми. Это связано с
	коммерческой составляющей, поскольку компании получают прибыль, реализуя устройства
	на собственных прошивках.
	
	По совокупности вышеописанных факторов был выбран другой подход, который не привязан
	к определённому протоколу. Суть данного подхода заключается в самостоятельной реализации
	криптографического уровня защиты и применении его поверх установленного соединения между
	управляющим устройством (хабом) и конечным устройством. В качестве конечного устройства
	в данной работе был выбран прототип умной лампочки.
	
	
	\section{Выбор компонентов и технологий для реализации}
	
	Обновлённые практические задачи были сформулированы следующим образом:
	
	\begin{itemize}
		\item выбрать микроконтроллер, который будет служить прототипом умного устройства,
		с возможностью его программирования и прошивки;
		\item установить соединение между управляющим и умным устройствами. Для простоты в качестве
		управляющего устройства в данной работе используется компьютер;
		\item разработать код (прошивку) для умного устройства (контроллера), а также клиентское
		приложение для управляющего устройства;
		\item реализовать защищённый обмен сообщениями с использование белорусского криптографического
		стандарта СТБ 34.101.77;
		\item модифицировать стандарт, изменив значения некоторых его параметров;
		\item оценить стойкость видоизменённого решения.
	\end{itemize}
	
	
	\section{Работа с микроконтроллером ESP8266}
	
	\section{Реализация и использование криптографического стандарта СТБ 34.101.77}
	
	\section{Модель прототипа}
	