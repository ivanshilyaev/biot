\chapter{Криптографические угрозы и атаки}

	\section{ZigBee}
	Поскольку во всех трёх решения (ZigBee, Z-Wave, Wi-Fi) в том или ином виде используется алгоритм 
	AES, шифрование данных оказывается весьма надёжным. Наиболее подверженным угрозам является
	этап подключения нового устройства в сеть. Рассмотрим этот этап для протокола ZigBee более подробно.
	
	\begin{enumerate}
		\item Устройство посылает в эфир запрос Beacon Request.
		\item В ответ ближайшие роутеры (или координатор в случае их отсутствия) отправляют Beacon Resposnse.
		\item Устройство выбирает роутер с наилучшими радио характеристиками и отправляет ему Association 
		Request.
		\item Роутер назначает новому устройству сетевой адрес и сообщает его в сообщении Association 
		Response.
		\item Роутер также информирует координатора о новом устройстве в сети.
		\item После этого координатор посылает на устройство network key (зашифрованный, как известно
		из предыдущего раздела с помощью pre-configured link key).
	\end{enumerate}

	Именно этот этап является самым уязвимым во всём сеансе сообщений. Основная задача злоумышленника
	$-$ перехватить network key, так как с его помощью открываются возможности для чтения показателей 
	датчиков, а также для отправки команд, которые будут считываться конечными устройствами. В случае,
	когда для присоединения новых устройств используется pre-configured global link key, заранее известный
	всем участникам, злоумышленнику достаточно прослушивать сеть в момент подключения. И когда
	координатор отправит новому устройству network key, зашифрованный на известном ключе, злоумышленнику
	останется только расшифровать его и использовать в своих целях.
	
	Можно считать, что эта уязвимость осознанно оставлена разработчиками протокола, так как временное 
	окно для подключения новых устройств обычно очень маленькое. При этом сохраняется совместимость
	между устройствами от различных производителей, а также удобство для потребителей, которым не
	нужно, например, самостоятельно записывать ключи шифрования на устройства. Однако уязвимость 
	остаётся, и у злоумышленника всегда будет шанс ей воспользоваться. Кроме того, он может попытаться
	заставить уже имеющиеся в сети устройства переподключаться, не дожидаясь тем самым момента для
	прослушивания сети, а инициируя его самостоятельно.
	
	Стоит отметить, что при использовании версии ZigBee 3.0 и pre-configured unique link key (ключа,
	уникального для каждого устройства), описанная выше уязвимость исчезает, и весь протокол
	становится безопасным. Но существует ещё много устройств, работающих на более старых версиях,
	для которых риск взлома по-прежнему остаётся.
	
	Несколько вариантов атак на предыдущие версии протокола ZigBee детально описаны в 
	\cite{zigbee-attacks}.


	\section{Z-Wave}
	
	% Убрать этот абзац позже:
	
	Многие устройства домашней автоматизации напрямую влияют на
	нашу безопасность. К таким устройствам можно отнести, например, автоматизированные дверные замки.
	В случае их компрометации последствия могут оказаться весьма неудачными. И такой случай
	имел место. В зашифрованных алгоритмом AES дверных замках на основе протокола Z-Wave была 
	обнаружена ранняя уязвимость. Использую её, злоумышленник получал возможность удалённо отпирать 
	двери без знания ключей шифрования. А после изменения ключей последующие сетевые сообщения, 
	например, <<дверь открыта>>, игнорировались установленным контроллером сети.
	Уязвимость не была связана с недостатком в спецификации Z-Wave, а являлась ошибкой 
	в реализации, допущенной производителем дверного замка.
	
	Безопасность протокола Z-Wave была улучшена в 2016 году с появлением спецификации Z-Wave S2 
	Security. Однако из-за обратной совместимости с предыдущей версией спецификации S0 устройства
	оказались по-прежнему уязвимы в процессе подключения. В Z-Wave S0 Security использовался статический
	первичный ключ, состоящий из 128 нулевых бит, из-за чего дальнейший взлом и управление устройствами 
	оказывались весьма тривиальными. В качестве улучшения в S2 для первичной выработки ключей 
	используется протокол Диффи-Хеллмана, а также дополнительно может потребоваться ввод 5-символьного
	кода устройства. Но в 2018 году была обнародована атака, которая позволяла понижать версию 
	спецификации с S2 до S0 и эксплуатировать старые уязвимости.
	
	% Выделить и переоформить атаку и шаги:
	
	Рассмотрим более подробно процесс понижения версии. Первые шаги при сопряжении устройства и 
	контроллера в спецификациях S0 и S2 аналогичны и заключаются в следующем:
	
	\begin{enumerate}
		\item На контроллере (управляющем устройстве) необходимо выбрать режим добавления нового
		устройства.
		\item Далее пользователь нажимает кнопку (или последовательность кнопок) на присоединяемом
		устройстве.
		\item После этого новое устройство посылает в сеть информацию о себе (Node info).
		\item Контроллер получает эту информацию и приступает к процессу выработки ключей.
	\end{enumerate}

	Единственным отличием в полезной нагрузке Node info для устройств на основе S2 Security является
	поддержка класса команды 0x9F – COMMAND\_ \newline \_CLASS\_SECURITY\_2. Стоит отметить, что
	вся информация в Node info является незашифрованной. Таким образом, активный атакующий может
	убрать соответствующую команду из Node info и отправить подделанные данные контроллеру. Контроллер
	посчитает, что устройство не поддерживает спецификацию S2 Security, и будет выбрана уязвимая к атакам
	спецификация S0. Справедливо заметить, что в этом случае контроллер должен выдать предупреждение
	пользователю об использовании более ранней версии, однако данное предупреждение, как правило,
	игнорируется. Подделанный Node info должен содержать тот же Home ID, что и у оригинального
	устройства. Поле Home ID не является константным, а генерируется каждый раз при добавлении или
	перезагрузке устройства. Это значит, что злоумышленник сперва должен завладеть Home ID. Эта
	задача выполнима и требует лишь пересчитывания длины сообщения и контрольной суммы после удаления 
	команды о поддержке S2.
	
	Резюмируя, можно сказать, что данная атака подчёркивает проблему множества протоколов, а именно
	проблему улучшения безопасности при необходимости поддержания старых устройств. Стоит отметить,
	что уже подключённые в сеть с использованием S2 Security и успешно функционирующие устройства 
	находятся в относительной безопасности, однако новые устройства остаются подверженными уязвимости.
	
	
	\section{Wi-Fi}
	
	% Выделить атаку, аналогично атаке на Z-Wave
	
	WPA2 являлся основным стандартом шифрования для Wi-Fi на протяжении 14 лет: с 2004 по 2018 год.
	Однако в 2016 году на WPA2 была разработана атака с переустановкой ключа (Key Reinstallation Attack, Krack).
	Krack $-$ это атака повторного воспроизведения \cite{cve-2017-13077}. Многократно сбрасывая одноразовый код, передаваемый 
	на третьем этапе установки соединения WPA2, злоумышленник может постепенно сопоставить зашифрованные 
	пакеты, сохранённые ранее, и узнать ключ шифрования.
	Уязвимость проявляется в самом стандарте Wi-Fi и не связана с ошибками реализации. В связи с этим любая
	корректная реализация WPA2 вероятнее всего является уязвимой. Уязвимость затрагивает основные 
	программные платформы.
	
	При подключении нового клиента к сети Wi-Fi с защитой по WPA2 общий ключ шифрования согласуется 
	за 4 этапа. Данный ключ служит для шифрования всех пакетов данных. Однако, из-за потери отдельных 
	сообщений точка доступа (роутер) может повторно отправлять сообщения третьего этапа до получения 
	подтверждение о его получении. Каждый раз при получении подобного сообщения клиент устанавливает 
	уже имеющийся ключ шифрования. На практике злоумышленник заставляет жертву выполнить переустановку
	ключа.
	
	Благодаря повторному использованию ключа шифрования появляется возможность воспроизведение пакетов, 
	расшифрования и подделки содержания. При определённых условиях злоумышленник способен осуществлять 
	атаки типа «человек посередине».
	
	С появлением стандарта WPA3 данная уязвимость была устранена благодаря введению SAE (Simultaneous 
	Authentication of Equals).
	
	
	\section{Матрица угроз}
	
	В данном разделе было решено сравнить устойчивость выбранных технологий к некоторому общему набору
	угроз в целях выявления наиболее криптостойкого решения.
	
	\begin{enumerate}
		\item Атака <<человек посередине>> (Man in the middle, MITM). Злоумышленник ретранслирует и 
		изменяет сообщения между участниками протокола.
		\item Атака повторного воспроизведения (Replay attack). Злоумышленник записывает сообщения
		из одного сеанса протокола в целях воспроизведения их в другом сеансе, выдавая себя за одного
		из участников.
		\item Защита от <<чтения назад>> (Perfect forward secrecy, PFS). Компрометация долговременных ключей 
		не должна приводить к компрометации предыдущих сеансовых ключей.
		\item Атака понижения версии (Downgrade attack). Злоумышленник принуждает устройства к
		использованию более старых и, соответственно, менее защищённых версий протокола.
	\end{enumerate}

	В отношении протокола ZigBee:
	
	\begin{itemize}
		\item При присоединении новых устройств здесь не используется протокол Диффи-Хеллмана
		(в отличие от, например, Z-Wave). Вследствие использования первоначального статического
		ключа, а также ключей, генерируемых координатором сети, данные первого же сообщения
		оказываются зашифрованными. Задача распределения ключей здесь решена другим способом
		(см. раздел про безопасность ZigBee).
		\item Для предотвращения от атак повторного воспроизведения в протоколе реализован
		специальный счётчик как часть процесса шифрования и аутентификации сообщений. С каждым
		новым пакетом данных значения счётчика увеличивается. Поскольку все пакеты зашифрованы,
		значение счётчика, как правило, не удаётся подменить. Однако в особых случая (компрометация
		сетевого ключа или долговременное использования сеансовых ключей без смены их координатором)
		теоретическая возможность данной атаки остаётся. Попытки использования этой возможности
		описаны в \cite{zigbee-attacks} и \cite{zigbee-security-analysis}.
		\item В ZigBee долговременным можно считать лишь pre-configured link key. Даже при выпуске 
		дополнительных ключей этот ключ продолжает храниться на устройстве на случай переприсоединения 
		или других нестандартных ситуаций. Теоретически его компрометация возможна только в случае
		физического доступа к устройству. Тогда появляется возможность получения других ключей,
		которые шифруются на нём, а вместе с ними и доступа к зашифрованным сообщения сеанса.
		В этом случае свойство PFS может не соблюдаться, однако даже здесь всё зависит от координатора
		сети и его действий по перевыпуску сеансовых ключей. В общем случае, при использовании последней
		версии протокола, компрометации pre-configured link key исключается.
		\item Версии ZigBee принципиально отличаются с позиции подхода к распределению ключей. В разных
		версиях pre-configured link key встраивается в устройства по-разному, о чём подробно шла речь
		в разделе про безопасность ZigBee. В связи с этим у злоумышленника отсутствует возможность
		понизить версию, если на устройствах уже используется более современная версия протокола.
	\end{itemize}

	Рассмотрим те же пункты касательно протокола Z-Wave:
	
	\begin{itemize}
		\item Как и в случае с ZigBee, процесс подключения новых устройств оказывается самым уязвимым.
		Для распределения ключей здесь, в отличие от ZigBee, используется протокол Диффи-Хеллмана. Для того,
		чтобы противостоять MITM-атаке, при реализации протокола используются различные модификации.
		Однако в \cite{formal-proof-of-z-wave-volnerability} приводится успешная атака несмотря на все методы 
		защиты. При этом атака может быть выполнена только в определённый временный промежуток и лишь
		при использовании одного из трёх возможных методов аутентификации в процессе подключения
		нового устройства. Поэтому есть большой шанс, что уязвимость будет устранена в ближайшем
		будущем.
		\item При использовании S2 Security атака повторного воспроизведения на протокол Z-Wave невозможна.
		\item Выше была подробна описаны атака понижения версии на протокол \newline Z-Wave в процессе подключения
		в сеть нового устройства. До тех пока, будет обеспечиваться обратная совместимость с предыдущими 
		уязвимыми версиями, вопрос защиты от атак подобного типа будет оставаться открытым.
	\end{itemize}

	Наконец посмотрим на те же угрозы применительно к Wi-Fi:
	
	\begin{itemize}
		\item В последних обновлениях WPA3 была представлена технология OCV (Operating Channel Validation),
		предотвращающая атаки типа <<Человек посередине>> и повторного воспроизведения. Основным же
		минусом является тот факт, что роутеры с поддержкой Wi-Fi 6 и WPA3 только начинают появляться
		на рынке, а уже имеющиеся решения весьма велики в цене.
		\item Протокол WPA2 спроектирован таким образом, что свойство PFS не может быть соблюдено.
		Здесь не используется криптография с открытым ключом, и у злоумышленника есть возможность
		сделать всё необходимое для чтения данных при компрометации долговременного ключа (пароля)
		в будущем. В WPA3 работают над невозможностью для злоумышленника записывать весь трафик
		между точкой доступа и устройством с целью расшифровки его в дальнейшем.
		\item Версия WPA3 тесно связана с WPA2 в контексте обратной совместимости, поэтому атаки,
		направленные на понижение версии, всё ещё имеют место быть. Единственным надёжным способом
		защиты от них может быть запрет на подключение в сеть WPA3 устройств с более слабым уровням
		защиты. Основная угроза, как и в других протоколах, появляется в момент рукопожатия (присоединения
		нового устройства в сеть).
	\end{itemize}

	В сводной сравнительной таблице символом <<$+$>> обозначено наличие защиты в данном протоколе
	от соответствующей угрозы, <<$-$>> $-$ отсутствие защиты, а <<$\sim$>> $-$ зависимость от версии
	протокола и прочих условий.
	
	\begin{table}[h]
		\centering
		\begin{tabular}{ | l | l | l | l | }
			\hline
			& ZigBee & Z-Wave & Wi-Fi \\ \hline
			<<Человек посередине>> & \multicolumn{1}{c|}{$+$} & \multicolumn{1}{c|}{$\sim$} & \multicolumn{1}{c|}{$+$} \\ \hline
			Атака повторного воспроизведения & \multicolumn{1}{c|}{$\sim$} &\multicolumn{1}{c|}{$+$} & \multicolumn{1}{c|}{$+$} \\ \hline
			Защита от <<чтения назад>> & \multicolumn{1}{c|}{$\sim$} & \multicolumn{1}{c|}{$\sim$} & \multicolumn{1}{c|}{$\sim$} \\ \hline
			Атака понижения версии & \multicolumn{1}{c|}{$+$} & \multicolumn{1}{c|}{$-$} & \multicolumn{1}{c|}{$\sim$} \\
			\hline
		\end{tabular}
		\caption{Матрица угроз IoT протоколов}
		\label{table-threat-matrix}
	\end{table}
	