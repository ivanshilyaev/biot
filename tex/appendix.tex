\newpage

\begin{flushright}{\bf \Large ПРИЛОЖЕНИЕ А}\end{flushright}
\begin{center}
{\bf Исходный код реализации алгоритмов bash-s и bash-f из стандарта СТБ 34.101.77 на языке 
	программирования C}
\end{center}

\addcontentsline{toc}{chapter}{Приложение А}

\begin{lstlisting}[language=C, label=bash-code]
#include <jni.h>
#include <inttypes.h>
#include <string.h>
#include "by_bsu_biot_library_LibraryNative.h"

typedef void *(*memset_t)(void *, int, size_t);

static volatile memset_t memset_func = memset;

void memory_clean(void *ptr, size_t len)
{
	memset_func(ptr, 0, len);
}

#define rol64(n, c) (((n) << (c)) | ((n) >> (64 - (c))))

#define bash_s(w0, w1, w2, m1, n1, m2, n2) \
do { \
	register uint64_t t0, t1, t2; \
	t0 = rol64(w0, m1); \
	w0 ^= w1 ^ w2; \
	t1 = w1 ^ rol64(w0, n1); \
	w1 = t0 ^ t1; \
	w2 ^= rol64(w2, m2) ^ rol64(t1, n2); \
	t1 = w0 | w2; \
	t2 = w0 & w1; \
	t0 = ~w2; \
	t0 |= w1; \
	w1 ^= t1; \
	w2 ^= t2; \
	w0 ^= t0; \
} while (0)

static const uint64_t __c1  = 0x3bf5080ac8ba94b1;
static const uint64_t __c2  = 0xc1d1659c1bbd92f6;
static const uint64_t __c3  = 0x60e8b2ce0ddec97b;
static const uint64_t __c4  = 0xec5fb8fe790fbc13;
static const uint64_t __c5  = 0xaa043de6436706a7;
static const uint64_t __c6  = 0x8929ff6a5e535bfd;
static const uint64_t __c7  = 0x98bf1e2c50c97550;
static const uint64_t __c8  = 0x4c5f8f162864baa8;
static const uint64_t __c9  = 0x262fc78b14325d54;
static const uint64_t __c10 = 0x1317e3c58a192eaa;
static const uint64_t __c11 = 0x098bf1e2c50c9755;
static const uint64_t __c12 = 0xd8ee19681d669304;
static const uint64_t __c13 = 0x6c770cb40eb34982;
static const uint64_t __c14 = 0x363b865a0759a4c1;
static const uint64_t __c15 = 0xc73622b47c4c0ace;
static const uint64_t __c16 = 0x639b115a3e260567;
static const uint64_t __c17 = 0xede6693460f3da1d;
static const uint64_t __c18 = 0xaad8d5034f9935a0;
static const uint64_t __c19 = 0x556c6a81a7cc9ad0;
static const uint64_t __c20 = 0x2ab63540d3e64d68;
static const uint64_t __c21 = 0x155b1aa069f326b4;
static const uint64_t __c22 = 0x0aad8d5034f9935a;
static const uint64_t __c23 = 0x0556c6a81a7cc9ad;
static const uint64_t __c24 = 0xde8082cd72debc78;

#define P0(x) x
#define P1(x) (((x) < 8) ? 8 + (((x) + 2 * ((x) & 1) + 7) % 8) : \
(((x) < 16) ? 8 + ((x) ^ 1) : (5 * (x) + 6) % 8))
#define P2(x) P1(P1(x))
#define P3(x) (8 * ((x) / 8) + (((x) % 8) + 4) % 8)
#define P4(x) P1(P3(x))
#define P5(x) P2(P3(x))

#define bash_r(s, p, p_next, i) \
do { \
	bash_s(s[p( 0)], s[p( 8)], s[p(16)],  8, 53, 14,  1); \
	bash_s(s[p( 1)], s[p( 9)], s[p(17)], 56, 51, 34,  7); \
	bash_s(s[p( 2)], s[p(10)], s[p(18)],  8, 37, 46, 49); \
	bash_s(s[p( 3)], s[p(11)], s[p(19)], 56,  3,  2, 23); \
	bash_s(s[p( 4)], s[p(12)], s[p(20)],  8, 21, 14, 33); \
	bash_s(s[p( 5)], s[p(13)], s[p(21)], 56, 19, 34, 39); \
	bash_s(s[p( 6)], s[p(14)], s[p(22)],  8,  5, 46, 17); \
	bash_s(s[p( 7)], s[p(15)], s[p(23)], 56, 35,  2, 55); \
	s[p_next(23)] ^= __c##i; \
} while (0)

JNIEXPORT jbyteArray JNICALL Java_by_bsu_biot_library_LibraryNative_bash_1f(JNIEnv *env, jclass thisClass, jbyteArray inJNIArray) {
	jbyte *inCArray = (*env)->GetByteArrayElements(env, inJNIArray, NULL);
	if (NULL == inCArray) return NULL;
	uint8_t arr[192];
	for (int i = 0; i < 192; i++)
	arr[i] = (uint8_t)inCArray[i];
	
	uint64_t s[24];
	memcpy(s, arr, 192);
	bash_r(s, P0, P1,  1);
	bash_r(s, P1, P2,  2);
	bash_r(s, P2, P3,  3);
	bash_r(s, P3, P4,  4);
	bash_r(s, P4, P5,  5);
	bash_r(s, P5, P0,  6);
	bash_r(s, P0, P1,  7);
	bash_r(s, P1, P2,  8);
	bash_r(s, P2, P3,  9);
	bash_r(s, P3, P4, 10);
	bash_r(s, P4, P5, 11);
	bash_r(s, P5, P0, 12);
	bash_r(s, P0, P1, 13);
	bash_r(s, P1, P2, 14);
	bash_r(s, P2, P3, 15);
	bash_r(s, P3, P4, 16);
	bash_r(s, P4, P5, 17);
	bash_r(s, P5, P0, 18);
	bash_r(s, P0, P1, 19);
	bash_r(s, P1, P2, 20);
	bash_r(s, P2, P3, 21);
	bash_r(s, P3, P4, 22);
	bash_r(s, P4, P5, 23);
	bash_r(s, P5, P0, 24);
	memcpy(arr, s, 192);
	memory_clean(s, 192);
	
	jbyte outCArray[192];
	for (int i = 0; i < 192; i++)
	outCArray[i] = (jbyte)arr[i];
	
	(*env)->ReleaseByteArrayElements(env, inJNIArray, inCArray, 0);
	
	jbyteArray outJNIArray = (*env)->NewByteArray(env, 192);
	if (NULL == outJNIArray) return NULL;
	(*env)->SetByteArrayRegion(env, outJNIArray, 0, 192, outCArray);
	return outJNIArray;
}
\end{lstlisting}


\newpage

\begin{flushright}{\bf \Large ПРИЛОЖЕНИЕ Б}\end{flushright}
\begin{center}
{\bf Исходный код реализации аутентифицированного шифрования из стандарта СТБ 34.101.77 
	на языке программирования Java}
\end{center}

\addcontentsline{toc}{chapter}{Приложение Б}

\begin{lstlisting}[language=Java,label=encr-code-java]
package by.bsu.biot.service;

import by.bsu.biot.dto.MachineDataType;
import by.bsu.biot.dto.EncryptionResult;
import by.bsu.biot.library.LibraryNative;
import org.apache.commons.lang3.ArrayUtils;
import org.springframework.stereotype.Service;

import java.util.ArrayList;
import java.util.Arrays;
import java.util.List;

@Service
public class EncryptionAndHashService {
	
	private static final int N = 1536;
	private static final int N_BYTES = 192;
	
	// уровень стойкости
	private int l;
	
	// ёмкость
	private int d;
	
	// длина буфера
	private int r;
	
	// текущее смещение в буфере
	private int pos;
	
	// состояние автомата
	private byte[] S;
	
	private byte[] I;
	
	public void init(int l, int d) {
		this.l = l;
		this.d = d;
		S = new byte[N_BYTES];
	}
	
	private void start(byte[] A, byte[] K) {
		// 1.
		if (K.length != 0) {
			r = N - l - d * l / 2;
		}
		// 2.
		else {
			r = N - 2 * d * l;
		}
		// 3.
		pos = 8 * (1 + A.length + K.length);
		// 4.
		S[0] = (byte) ((8 * A.length / 2 + 8 * K.length / 32) % (Math.pow(2, 8)));
		System.arraycopy(A, 0, S, 1, A.length);
		System.arraycopy(K, 0, S, 1 + A.length, K.length);
		// 5.
		for (int i = pos / 8; i < 1472 / 8; ++i) {
			S[i] = 0;
		}
		// 6.
		S[1472 / 8] = (byte) ((l / 4 + d) % (Math.pow(2, 64)));
		for (int i = 1472 / 8 + 1; i < N_BYTES; ++i) {
			S[i] = 0;
		}
	}
	
	private void commit(MachineDataType type) {
		// 1.
		S[pos / 8] = (byte) (S[pos / 8] ^ Byte.parseByte(type.code, 2));
		// 2.
		S[r / 8] = (byte) (S[r / 8] ^ 128); // 1000 0000
		// 3.
		S = LibraryNative.bash_f(S);
		// 4.
		pos = 0;
	}
	
	private void absorb(byte[] X) {
		// 1.
		commit(MachineDataType.DATA);
		// 2.
		byte[][] XSplit = split(X, r);
		// 3.
		for (byte[] Xi : XSplit) {
			// 3.1
			pos = Xi.length * 8;
			// 3.2
			for (int i = 0; i < pos / 8; ++i) {
				S[i] = (byte) (S[i] ^ Xi[i]);
			}
			// 3.3
			if (pos == r) {
				S = LibraryNative.bash_f(S);
				pos = 0;
			}
		}
	}
	
	private byte[] squeeze(int n) {
		// 1.
		commit(MachineDataType.OUT);
		// 2.
		byte[] Y = new byte[0];
		// 3.
		while (8 * Y.length + r <= n) {
			// 3.1
			Y = ArrayUtils.addAll(Y, ArrayUtils.subarray(S, 0, r / 8));
			// 3.2
			S = LibraryNative.bash_f(S);
		}
		// 4.
		pos = n - 8 * Y.length;
		// 5.
		Y = ArrayUtils.addAll(Y, ArrayUtils.subarray(S, 0, pos / 8));
		// 6.
		return Y;
	}
	
	private byte[] encrypt(byte[] X) {
		// 1.
		commit(MachineDataType.TEXT);
		// 2.
		byte[][] XSplit = split(X, r);
		// 3.
		byte[] Y = new byte[0];
		// 4.
		for (byte[] Xi : XSplit) {
			// 4.1
			pos = Xi.length * 8;
			// 4.2
			for (int i = 0; i < pos / 8; ++i) {
				S[i] = (byte) (S[i] ^ Xi[i]);
			}
			// 4.3
			Y = ArrayUtils.addAll(Y, ArrayUtils.subarray(S, 0, pos / 8));
			// 4.4
			if (pos == r) {
				S = LibraryNative.bash_f(S);
				pos = 0;
			}
		}
		// 5.
		return Y;
	}
	
	private byte[] decrypt(byte[] Y) {
		// 1.
		commit(MachineDataType.TEXT);
		// 2.
		byte[][] YSplit = split(Y, r);
		// 3.
		byte[] X = new byte[0];
		// 4.
		for (byte[] Yi : YSplit) {
			// 4.1
			pos = Yi.length * 8;
			// 4.2
			byte[] tempS = ArrayUtils.subarray(S, 0, pos / 8);
			for (int i = 0; i < pos / 8; ++i) {
				tempS[i] = (byte) (tempS[i] ^ Yi[i]);
			}
			X = ArrayUtils.addAll(X, tempS);
			// 4.3
			System.arraycopy(Yi, 0, S, 0, pos / 8);
			// 4.4
			if (pos == r) {
				S = LibraryNative.bash_f(S);
				pos = 0;
			}
		}
		// 5.
		return X;
	}
	
	/**
	* @param A - анонс
	* @param K - ключ
	* @param I - ассоциированные данные
	* @param X - сообщение
	* @return зашифрованное сообщение Y и имитовставка T
	*/
	public EncryptionResult authEncrypt(byte[] A, byte[] K, byte[] I, byte[] X) {
		// 1.
		start(A, K);
		// 2.1
		absorb(I);
		// 2.2
		byte[] Y = encrypt(X);
		// 2.3
		byte[] T = squeeze(l);
		// 2.4
		return EncryptionResult.builder()
		.Y(Y)
		.T(T)
		.build();
	}
	
	/**
	 * @param A - анонс
	 * @param K - ключ
     * @param I - ассоциированные данные
	 * @param Y - зашифрованное сообщение
	 * @param T - имитовставка
	 * @return расшифрованное сообщение X
	 */
	public byte[] authDecrypt(byte[] A, byte[] K, byte[] I, byte[] Y, byte[] T) {
		// 1.
		start(A, K);
		// 2.1
		absorb(I);
		// 2.2
		byte[] X = decrypt(Y);
		// 2.3
		if (!Arrays.equals(T, squeeze(l))) {
			return null;
		}
		return X;
	}
	
	private static byte[][] split(byte[] bytes, int r) {
		if (bytes.length * 8 <= r) {
			return new byte[][]{bytes};
		}
		List<byte[]> result = new ArrayList<>();
		int tempPos = 0;
		while (tempPos * 8 + r < bytes.length * 8) {
			result.add(ArrayUtils.subarray(bytes, tempPos, tempPos + r / 8));
			tempPos += r / 8;
		}
		result.add(ArrayUtils.subarray(bytes, tempPos, bytes.length));
		
		return result.toArray(new byte[0][]);
	}
}

\end{lstlisting}


\newpage

\begin{flushright}{\bf \Large ПРИЛОЖЕНИЕ В}\end{flushright}
\begin{center}
	{\bf Исходный код реализации аутентифицированного шифрования из стандарта СТБ 34.101.77 
		на языке программирования C++}
\end{center}

\addcontentsline{toc}{chapter}{Приложение В}

\begin{lstlisting}[language=C++,label=encr-code-cpp]
#include <iostream>
#include <cmath>
#include <string>
#include <sstream>
#include <iomanip>
#include <algorithm>
extern "C" {
	#include "library.h"
}
using namespace std;

size_t decode(uint8_t dest[], size_t count, const char *src) {
	char buf[3];
	size_t i;
	for (i = 0; i < count && *src; i++) {
		buf[0] = *src++;
		buf[1] = '\0';
		if (*src) {
			buf[1] = *src++;
			buf[2] = '\0';
		}
		if (sscanf(buf, "%hhx", &dest[i]) != 1)
		break;
	}
	return i;
}

void reverseAndDecode(uint8_t array[], string s) {
	string temp;
	for (int i = (int) (s.length() - 1); i >= 0; i -= 2) {
		temp += s[i-1];
		temp += s[i];
	}
	decode(array, temp.length(), temp.c_str());
	// reverse
	for (int i = 0; i < s.length() / 4; ++i) {
		uint8_t temp = array[i];
		array[i] = array[s.length() / 2 - 1 - i];
		array[s.length() / 2 - 1 - i] = temp;
	}
}

string encode(const uint8_t v[], const size_t s) {
	stringstream ss;
	
	ss << hex << setfill('0');
	
	for (int i = 0; i < s; i++) {
		ss << hex << setw(2) << static_cast<int>(v[i]);
	}
	
	return ss.str();
}

size_t l;
size_t d;
size_t r; // buf_len
size_t pos;
uint8_t S[192] = {0};

#define BASH_PRG_NULL       1 // 0x01, 000000 01 */
#define BASH_PRG_KEY        5 // 0x05, 000001 01 */
#define BASH_PRG_DATA       9 // 0x09, 000010 01 */
#define BASH_PRG_TEXT      13 // 0x0D, 000011 01 */
#define BASH_PRG_OUT       17 // 0x11, 000100 01 */

void start(uint8_t A[], size_t A_len, uint8_t K[], size_t K_len) {
	// 1.
	if (K_len != 0) {
		r = 1536 - l - d * l / 2;
	}
	// 2.
	else {
		r = 1536 - 2 * d * l;
	}
	// 3.
	pos = 8 * (1 + A_len + K_len);
	// 4.
	S[0] = (8 * A_len / 2 + 8 * K_len / 32) % (size_t)(pow(2, 8));
	for (int i = 0; i < A_len; ++i) {
		S[1 + i] = A[i];
	}
	for (int i = 0; i < K_len; ++i) {
		S[1 + A_len + i] = K[i];
	}
	// 5.
	for (int i = (int) pos / 8; i < 1472 / 8; ++i) {
		S[i] = 0;
	}
	// 6.
	S[1472 / 8] = (l / 4 + d) % (size_t)(pow(2, 8));
	for (int i = 1472 / 8 + 1; i < 192; ++i) {
		S[i] = 0;
	}
}

void commit(uint8_t type) {
	// 1.
	S[pos / 8] = S[pos / 8] ^ type;
	// 2.
	S[r / 8] = S[r / 8] ^ 128; // 1000 0000
	// 3.
	bash_f(S);
	// 4.
	pos = 0;
}

void absorb(uint8_t X[], size_t X_len) {
	// 1.
	commit(BASH_PRG_DATA);
	// 2-3.
	int tempPos = 0;
	while (8 * tempPos + r < 8 * X_len) {
		// 3.1
		pos = r;
		// 3.2
		for (int i = 0; i < pos / 8; ++i) {
			S[i] = S[i] ^ X[tempPos + i];
		}
		// 3.3
		bash_f(S);
		pos = 0;
		
		tempPos += r / 8;
	}
	// finish steps 2 & 3
	pos = 8 * (X_len - tempPos);
	for (int i = 0; i < pos  / 8; ++i) {
		S[i] = S[i] ^ X[tempPos + i];
	}
}

void squeeze(uint8_t Y[], size_t n) {
	// 1.
	commit(BASH_PRG_OUT);
	// 2-3.
	int tempPos = 0;
	while (8 * tempPos + r <= n) {
		// 3.1
		for (int i = 0; i < r / 8; ++i) {
			Y[tempPos + i] = S[i];
		}
		tempPos += r / 8;
		// 3.2
		bash_f(S);
	}
	// 4.
	pos = n - 8 * tempPos;
	// 5.
	for (int i = 0; i < pos / 8; ++i) {
		Y[tempPos + i] = S[i];
	}
}

void encrypt(uint8_t X[], size_t X_len, uint8_t Y[]) {
	// 1.
	commit(BASH_PRG_TEXT);
	// 2-4.
	int tempPos = 0;
	while (8 * tempPos + r < 8 * X_len) {
		// 4.1
		pos = r;
		// 4.2
		for (int i = 0; i < pos / 8; ++i) {
			S[i] = S[i] ^ X[tempPos + i];
		}
		// 4.3
		for (int i = 0; i < pos / 8; ++i) {
			Y[tempPos + i] = S[i];
		}
		// 4.4
		bash_f(S);
		pos = 0;
		
		tempPos += r / 8;
	}
	// finish steps 2 & 4
	pos = 8 * (X_len - tempPos);
	for (int i = 0; i < pos / 8; ++i) {
		S[i] = S[i] ^ X[tempPos + i];
	}
	for (int i = 0; i < pos / 8; ++i) {
		Y[tempPos + i] = S[i];
	}
}

void decrypt(uint8_t Y[], size_t Y_len, uint8_t X[]) {
	// 1.
	commit(BASH_PRG_TEXT);
	// 2-4.
	int tempPos = 0;
	while (8 * tempPos + r < 8 * Y_len) {
		// 4.1
		pos = r;
		// 4.2
		for (int i = 0; i < pos / 8; ++i) {
			X[tempPos + i] = S[i] ^ Y[tempPos + i];
		}
		// 4.3
		for (int i = 0; i < pos / 8; ++i) {
			S[i] = Y[tempPos + i];
		}
		// 4.4
		bash_f(S);
		pos = 0;
		
		tempPos += r / 8;
	}
	// finish steps 2 & 4
	pos = 8 * (Y_len - tempPos);
	for (int i = 0; i < pos / 8; ++i) {
		X[tempPos + i] = S[i] ^ Y[tempPos + i];
	}
	for (int i = 0; i < pos / 8; ++i) {
		S[i] = Y[tempPos + i];
	}
}

void authEncrypt(size_t varL, size_t varD,
uint8_t A[], size_t A_len,
uint8_t K[], size_t K_len,
uint8_t I[], size_t I_len,
uint8_t X[], size_t X_len,
uint8_t Y[], uint8_t T[]) {
	l = varL;
	d = varD;
	// 1.
	start(A, A_len, K, K_len);
	// 2.1
	absorb(I, I_len);
	// 2.2
	encrypt(X, X_len, Y);
	// 2.3
	squeeze(T, l);
}

void authDecrypt(size_t varL, size_t varD,
uint8_t A[], size_t A_len,
uint8_t K[], size_t K_len,
uint8_t I[], size_t I_len,
uint8_t Y[], size_t Y_len,
uint8_t X[], uint8_t T[],
bool error) {
	l = varL;
	d = varD;
	// 1.
	start(A, A_len, K, K_len);
	// 2.1
	absorb(I, I_len);
	// 2.2
	decrypt(Y, Y_len, X);
	// 2.3
	uint8_t tempT[l];
	squeeze(tempT, l);
	for (int i = 0; i < 32; ++i) {
		if (tempT[i] != T[i]) {
			error = true;
			break;
		}
	}
}

\end{lstlisting}


\newpage

\begin{flushright}{\bf \Large ПРИЛОЖЕНИЕ Г}\end{flushright}
\begin{center}
{\bf Исходный код реализации сервиса по отправке команд на умное устройство 
	на языке программирования Java}
\end{center}

\addcontentsline{toc}{chapter}{Приложение Г}

\begin{lstlisting}[language=Java,label=client-code]
package by.bsu.biot.service;

import by.bsu.biot.dto.EncryptionResult;
import by.bsu.biot.util.HexEncoder;
import lombok.RequiredArgsConstructor;
import lombok.extern.slf4j.Slf4j;
import okhttp3.FormBody;
import okhttp3.OkHttpClient;
import okhttp3.Request;
import okhttp3.Response;
import org.springframework.beans.factory.annotation.Value;
import org.springframework.stereotype.Service;

import javax.annotation.PostConstruct;
import java.io.IOException;
import java.nio.charset.StandardCharsets;
import java.util.Base64;
import java.util.concurrent.TimeUnit;

@Service
@RequiredArgsConstructor
@Slf4j
public class ClientService {
	
	private final EncryptionAndHashService encryptionService;
	
	@Value("${biot.encryption.enabled}")
	private boolean encryptionEnabled;
	
	private String encryptionKey;
	
	private int messageCount;
	
	private static final String LUMP_STATIC_IP_ADDRESS = "192.168.100.93";
	
	private static final String LED_PATH = "/led";
	
	private static final String KEY_PATH = "/key";
	
	private static final int HTTP_SUCCESS_CODE = 200;
	
	private static final OkHttpClient client = new OkHttpClient().newBuilder()
	.connectTimeout(10, TimeUnit.SECONDS)
	.readTimeout(30, TimeUnit.SECONDS)
	.build();
	
	@PostConstruct
	public void init() {
		int l = 256;
		int d = 1;
		byte[] I = new byte[0];
		encryptionService.init(l, d, I);
		messageCount = 0;
		
		encryptionKey = HexEncoder.generateRandomHexString(System.getenv("INITIAL_ENCRYPTION_KEY").length());
		log.info("encryption key: " + encryptionKey);
	}
	
	public void turnOn() throws IOException {
		log.info("onn command sent");
		if (encryptionEnabled) {
			sendEncryptedMessage("onn", encryptionKey, LED_PATH);
		} else {
			sendPostRequest("onn", "", LED_PATH);
		}
	}
	
	public void turnOff() throws IOException {
		log.info("off command sent");
		if (encryptionEnabled) {
			sendEncryptedMessage("off", encryptionKey, LED_PATH);
		} else {
			sendPostRequest("off", "", LED_PATH);
		}
	}
	
	public void sendEncryptionKey() throws IOException {
		sendEncryptedMessage(encryptionKey, System.getenv("INITIAL_ENCRYPTION_KEY"), KEY_PATH);
	}
	
	private void sendEncryptedMessage(String message, String key, String path) throws IOException {
		byte[] A = String.valueOf(++messageCount).getBytes();
		byte[] K = HexEncoder.decode(key);
		byte[] X = message.getBytes(StandardCharsets.UTF_8);
		
		EncryptionResult encryptionResult = encryptionService.authEncrypt(A, K, X);
		
		log.info("encrypted message: " + HexEncoder.encode(encryptionResult.getY()));
		log.info("mac: " + HexEncoder.encode(encryptionResult.getT()));
		String encryptedMessage = new String(
		Base64.getEncoder().encode(HexEncoder.encode(encryptionResult.getY()).getBytes()));
		String mac = new String(
		Base64.getEncoder().encode(HexEncoder.encode(encryptionResult.getT()).getBytes()));
		
		Response response = sendPostRequest(encryptedMessage, mac, path);
		if (response.code() == HTTP_SUCCESS_CODE) {
			log.info("message has been successfully processed");
			log.info("count: " + ++messageCount);
		}
	}
	
	private Response sendPostRequest(String param1, String param2, String path) throws IOException {
		FormBody body = new FormBody.Builder()
		.add("param1", param1)
		.add("param2", param2)
		.build();
		Request request = new Request.Builder()
		.url("http://" + LUMP_STATIC_IP_ADDRESS + path)
		.post(body)
		.build();
		
		return client.newCall(request).execute();
	}
}

\end{lstlisting}


\newpage

\begin{flushright}{\bf \Large ПРИЛОЖЕНИЕ Д}\end{flushright}
\begin{center}
	{\bf Исходный код реализации прошивки для микроконтроллера ESP8266 
		на языке программирования C++}
\end{center}

\addcontentsline{toc}{chapter}{Приложение Д}

\begin{lstlisting}[language=C++,label=esp-code]
#include <ESP8266WiFi.h>
#include <DNSServer.h>
#include <ESP8266WebServer.h>
#include <WiFiManager.h>
#include "standard77.hpp"
#include "base64.hpp"
#include "secrets.h"

const char* PARAM_INPUT_1 = "param1";
const char* PARAM_INPUT_2 = "param2";
const char* ERROR_MESSAGE = "error";
const char* ENCODED_ON_COMMAND = "6f6e6e";
const char* ENCODED_OFF_COMMAND = "6f6666";

ESP8266WebServer server(80);

int LED = D1;

string ENCRYPTION_KEY;

int messageCount;

void healthCheck() {
	server.send(200, "text/plain", "Ok");
}

string decrypt(string key) {
	if (server.hasArg(PARAM_INPUT_1) && server.hasArg(PARAM_INPUT_2)) {
		string encodedMessage = server.arg(PARAM_INPUT_1).c_str();
		string encodedMac = server.arg(PARAM_INPUT_2).c_str();
		
		uint8_t hexMessageArray[BASE64::decodeLength(encodedMessage.c_str())];
		BASE64::decode(encodedMessage.c_str(), hexMessageArray);
		string hexMessage = reinterpret_cast<char *>(hexMessageArray);
		
		Serial.println("encrypted message:");
		Serial.println(hexMessage.c_str());
		
		uint8_t hexMacArray[32];
		BASE64::decode(encodedMac.c_str(), hexMacArray);
		string hexMac = reinterpret_cast<char *>(hexMacArray); 
		
		Serial.println("mac:");
		Serial.println(hexMac.c_str());
		
		size_t l = 256;
		size_t d = 1;
		
		string countString = std::to_string(++messageCount);
		uint8_t A[countString.length()];
		Serial.println("count:");
		Serial.println(messageCount);
		char aChar[countString.length()];
		strcpy(aChar, countString.c_str());
		for (int i = 0; i < countString.length(); ++i) {
			A[i] = (uint8_t)aChar[i];
		}
		uint8_t K[32];
		reverseAndDecode(K, key);
		uint8_t I[0];
		
		size_t Y_len = hexMessage.length() / 2;
		uint8_t Y[Y_len];
		decode(Y, Y_len, hexMessage.c_str());
		
		uint8_t X[Y_len];
		
		size_t T_len = 16;
		uint8_t T[T_len];
		decode(T, T_len, hexMac.c_str());
		
		bool error = false;
		authDecrypt(l, d, A, countString.length(), K, 32, I, 0, Y, Y_len, X, T, error);
		if (!error) {
			return encode(X, Y_len);
		}
	}
	return ERROR_MESSAGE;
}

void handleKey() {
	ENCRYPTION_KEY.clear();
	ENCRYPTION_KEY = decrypt(INITIAL_ENCRYPTION_KEY);
	if (ENCRYPTION_KEY != ERROR_MESSAGE) {
		Serial.println("encryption key:");
		Serial.println(ENCRYPTION_KEY.c_str());
		++messageCount;
		server.send(200);
	}
}

void handleBody() {
	if (ENCRYPTIN_ENABLED) {
		string state = decrypt(ENCRYPTION_KEY);
		if (state != ERROR_MESSAGE) {
			if (state == ENCODED_ON_COMMAND) {
				Serial.println("encrypted command received: onn");
				digitalWrite(LED, HIGH);
				++messageCount;
				server.send(200);
			}
			else if (state == ENCODED_OFF_COMMAND) {
				Serial.println("encrypted command received: off");
				digitalWrite(LED, LOW);
				++messageCount;
				server.send(200);
			}
		}
	}
	else {
		if (server.hasArg(PARAM_INPUT_1)) {
			String state = server.arg(PARAM_INPUT_1);
			if (state == "onn") {
				Serial.println("command received: onn");
				digitalWrite(LED, HIGH);
				++messageCount;
				server.send(200);
			}
			else if (state == "off") {
				Serial.println("command received: off");
				digitalWrite(LED, LOW);
				++messageCount;
				server.send(200);
			}
		}
	}
}

void setup() {
	Serial.begin(115200);
	WiFiManager wifiManager;
	// wifiManager.resetSettings();
	
	// wifiManager.startConfigPortal(LIGHT_SSID, LIGHT_PASSWORD);
	wifiManager.autoConnect(LIGHT_SSID, LIGHT_PASSWORD);
	
	pinMode(LED, OUTPUT);
	digitalWrite(LED, HIGH);
	messageCount = 0;
	
	server.on("/health", HTTP_GET, healthCheck); // health check request
	server.on("/key", HTTP_POST, handleKey); // encryption key
	server.on("/led", HTTP_POST, handleBody); // led commands
	server.begin();
}

void loop() {
	server.handleClient();
}

\end{lstlisting}
